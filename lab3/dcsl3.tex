\documentclass[a4paper,14pt]{extarticle}

\usepackage[T2A]{fontenc}
\usepackage[utf8]{inputenc}
\usepackage[english, russian]{babel}

\usepackage[left=30mm, right=10mm, top=20mm, bottom=20mm]{geometry}

\usepackage{tempora}
\usepackage{setspace}
\onehalfspacing

\usepackage{titlesec}
\titleformat{\section}[block]{\bfseries\centering\MakeUppercase}{\thesection.}{1em}{}
\titleformat{\subsection}[block]{\bfseries}{\thesubsection.}{1em}{}
\titleformat{\subsubsection}[block]{\bfseries}{\thesubsubsection.}{1em}{}

\renewcommand{\contentsname}{\hfill \textbf{СОДЕРЖАНИЕ} \hfill\null}

\usepackage{indentfirst}
\setlength{\parindent}{1.25cm}

\usepackage{amsmath, amsfonts, amssymb}
\usepackage{graphicx}
\usepackage{caption}
\usepackage{subcaption}
\usepackage{float}
\usepackage{tikz}
\usetikzlibrary{patterns}
\usepackage{cmap}
\usepackage{hyperref}
\usepackage{xcolor}
\usepackage{listings}

\definecolor{LightGray}{gray}{0.7}

\lstdefinestyle{code}{
    language=Python, % change if needed
    basicstyle=\small\ttfamily,
    numbers=left,
    numberstyle=\small\color{LightGray},
    stepnumber=1,
    numbersep=5pt,
    backgroundcolor=\color{white},
    showspaces=false,
    showstringspaces=false,
    showtabs=false,
    tabsize=4,
    captionpos=b,
    breaklines=true,
    breakatwhitespace=false,
    frame=single,
    rulecolor=\color{LightGray},
    linewidth=\linewidth,
    keywordstyle=\color{blue}\bfseries,
    commentstyle=\color{green!40!black},
    stringstyle=\color{violet},
    escapeinside={\%*}{*)},
    xleftmargin=10pt,
    xrightmargin=10pt,
    framexleftmargin=0pt,
    framexrightmargin=0pt
}
\lstset{style=code}

\hypersetup{
    colorlinks=true,
    linkcolor=blue,
    filecolor=magenta,
    urlcolor=cyan,
    pdftitle={ncs1},
    pdfauthor={Rumyantsev Alexey},
    pdfsubject={control},
    pdfkeywords={LaTeX, PDF},
    pdfpagemode=FullScreen,
}

\graphicspath{{src/images/}}

\begin{document}

\begin{titlepage}
    \begin{center}
        МИНИСТЕРСТВО НАУКИ И ВЫСШЕГО ОБРАЗОВАНИЯ РОССИЙСКОЙ ФЕДЕРАЦИИ\\
        \vspace*{2.5mm}
        Федеральное государственное автономное образовательное учреждение высшего образования
        «НАЦИОНАЛЬНЫЙ ИССЛЕДОВАТЕЛЬСКИЙ УНИВЕРСИТЕТ ИТМО»\\
        \vspace*{2.5mm}
        \textbf{ФАКУЛЬТЕТ СИСТЕМ УПРАВЛЕНИЯ И РОБОТОТЕХНИКИ}
        \vfill

        {\large ОТЧЕТ ПО ЛАБОРАТОРНОЙ РАБОТЕ №3}\\
        {\large по дисциплине}\\
        {\large\bfseries «ДИСКРЕТНЫЕ СИСТЕМЫ УПРАВЛЕНИЯ»}\\
        {\large на тему}\\
        {\large\bfseries «ДИСКРЕТНЫЙ ПИД РЕГУЛЯТОР»}\\
        Вариант 20
        \vfill

        \begin{flushright}
            Выполнил: студент гр. R3441\\
            Румянцев А. А.\medskip\\

            Проверил: преподаватель\\
            Краснов А. Ю.
        \end{flushright}
        \vfill

        Санкт-Петербург\\
        2025
    \end{center}
\end{titlepage}

\setcounter{page}{2}
\tableofcontents
\newpage

\section{Исходные данные}
Исходные данные варианта 20:
\begin{table}[h!]
\centering
\begin{tabular}{|c|c|}
\hline
$T_1$ & $T_2$\\ \hline
1.35 & 1.2 \\ \hline
\end{tabular}
\end{table}


$T_1, T_2$ -- постоянные времени ОУ.


\section{Выполнение работы}
\subsection{Модель системы}
Модель системы в Simulink:
\begin{figure}[H]
    \centering
    \includegraphics[scale=0.25]{sch1.png}
    \caption{Схема моделирования цифровой САУ температуры}
    \label{fig:sch1}
\end{figure}


\subsection{Значения параметров схемы}
Установим значение периода дискретизации
в модели экстраполятора нулевого порядка
$T=T_1/2\approx0.675$.


Рассчитаем значения полюсов
приведенной непрерывной части:
$$
z_1=d_1=e^{-\frac{T}{T_1}}\approx0.607,
$$
$$
z_2=d_2=e^{-\frac{T}{T_2}}\approx0.57
$$


Посчитаем значения коэффициентов полинома
дискретного регулятора:
$$
\text{num}(z)=z^2+\left( -d_1-d_2 \right)z+d_1d_2=z^2-1.176z+0.346
$$


Установим полученные коэффициенты
полинома в блок Discrete Transfer Fcn и
значения постоянных времени $T_1,T_2$
в модель ОУ.


\subsection{Значение коэффициента передачи регулятора}
Проверим выход системы при значениях коэффициента передачи регулятора $q_0=0.0007,q_0=0.0008$
при задающем воздействии $r(t)=1$:
\begin{figure}[H]
    \centering
    \includegraphics[scale=0.9]{q0_07_y.png}
    \caption{Выход системы при $q_0=0.0007,r(t)=1$}
    \label{fig:q0_07_y}
\end{figure}
% \begin{figure}[H]
%     \centering
%     \includegraphics[scale=0.9]{q0_07_u.png}
%     \caption{Дискретный ПИД регулятор при $q_0=0.0007,r(t)=1$}
%     \label{fig:q0_07_u}
% \end{figure}
\begin{figure}[H]
    \centering
    \includegraphics[scale=0.9]{q0_08_y.png}
    \caption{Выход системы при $q_0=0.0008,r(t)=1$}
    \label{fig:q0_08_y}
\end{figure}
% \begin{figure}[H]
%     \centering
%     \includegraphics[scale=0.9]{q0_08_u.png}
%     \caption{Дискретный ПИД регулятор при $q_0=0.0008,r(t)=1$}
%     \label{fig:q0_08_u}
% \end{figure}
\begin{figure}[H]
    \centering
    \includegraphics[scale=0.9]{1_r_1.png}
    \caption{Задающее воздействие $r(t)=1$}
    \label{fig:1_r_1}
\end{figure}


При $q_0=0.0008$ система устойчива и имеет слабоколебательные
переходные процессы, при $q_0<0.0008$ сходится монотонно.


\subsection{Процессы на выходе дискретного регулятора и системы}
Исследуем ступенчатое изменение задающего воздействия:
\begin{figure}[H]
    \centering
    \includegraphics[scale=0.9]{1_r_step_10_0_1_0.png}
    \caption{Задающее воздействие $r_1(t)$}
    \label{fig:1_r_step_10_0_1_0}
\end{figure}
\begin{figure}[H]
    \centering
    \includegraphics[scale=0.9]{1_y_step_10_0_1_0.png}
    \caption{Выход системы при $r_1(t)$}
    \label{fig:1_y_step_10_0_1_0}
\end{figure}
\begin{figure}[H]
    \centering
    \includegraphics[scale=0.9]{1_u_step_10_0_1_0.png}
    \caption{Дискретный ПИД регулятор при $r_1(t)$}
    \label{fig:1_u_step_10_0_1_0}
\end{figure}
\begin{figure}[H]
    \centering
    \includegraphics[scale=0.9]{1_e_step_10_0_1_0.png}
    \caption{Ошибка $e=r_1(t)-y(t)$}
    \label{fig:1_e_step_10_0_1_0}
\end{figure}


\begin{figure}[H]
    \centering
    \includegraphics[scale=0.9]{1_r_step_10_3_1_0.png}
    \caption{Задающее воздействие $r_2(t)$}
    \label{fig:1_r_step_10_3_1_0}
\end{figure}
\begin{figure}[H]
    \centering
    \includegraphics[scale=0.9]{1_y_step_10_3_1_0.png}
    \caption{Выход системы при $r_2(t)$}
    \label{fig:1_y_step_10_3_1_0}
\end{figure}
\begin{figure}[H]
    \centering
    \includegraphics[scale=0.9]{1_u_step_10_3_1_0.png}
    \caption{Дискретный ПИД регулятор при $r_2(t)$}
    \label{fig:1_u_step_10_3_1_0}
\end{figure}
\begin{figure}[H]
    \centering
    \includegraphics[scale=0.9]{1_e_step_10_3_1_0.png}
    \caption{Ошибка $e=r_2(t)-y(t)$}
    \label{fig:1_e_step_10_3_1_0}
\end{figure}


\begin{figure}[H]
    \centering
    \includegraphics[scale=0.9]{1_r_step_10_m1_1_0.png}
    \caption{Задающее воздействие $r_3(t)$}
    \label{fig:1_r_step_10_m1_1_0}
\end{figure}
\begin{figure}[H]
    \centering
    \includegraphics[scale=0.9]{1_y_step_10_m1_1_0.png}
    \caption{Выход системы при $r_3(t)$}
    \label{fig:1_y_step_10_m1_1_0}
\end{figure}
\begin{figure}[H]
    \centering
    \includegraphics[scale=0.9]{1_u_step_10_m1_1_0.png}
    \caption{Дискретный ПИД регулятор при $r_3(t)$}
    \label{fig:1_u_step_10_m1_1_0}
\end{figure}
\begin{figure}[H]
    \centering
    \includegraphics[scale=0.9]{1_e_step_10_m1_1_0.png}
    \caption{Ошибка $e=r_3(t)-y(t)$}
    \label{fig:1_e_step_10_m1_1_0}
\end{figure}


Положим $r(t)=1$ (см. рис. (\ref{fig:1_r_1})).


Исследуем ступенчатое изменение возмущающего воздействия $d(t)$
полагая, что $d_i(t)=r_i(t)$ (см. рис. (\ref{fig:1_r_step_10_0_1_0}), (\ref{fig:1_r_step_10_3_1_0}), (\ref{fig:1_r_step_10_m1_1_0})):
\begin{figure}[H]
    \centering
    \includegraphics[scale=0.9]{1_y_stepd_10_0_1_0.png}
    \caption{Выход системы при $d_1(t)=r_1(t),r(t)=1$}
    \label{fig:1_y_stepd_10_0_1_0}
\end{figure}
\begin{figure}[H]
    \centering
    \includegraphics[scale=0.9]{1_u_stepd_10_0_1_0.png}
    \caption{Дискретный ПИД регулятор при $d_1(t)=r_1(t),r(t)=1$}
    \label{fig:1_u_stepd_10_0_1_0}
\end{figure}
\begin{figure}[H]
    \centering
    \includegraphics[scale=0.9]{1_e_stepd_10_0_1_0.png}
    \caption{Ошибка $e=r_1(t)-y(t),r(t)=1$}
    \label{fig:1_e_stepd_10_0_1_0}
\end{figure}


\begin{figure}[H]
    \centering
    \includegraphics[scale=0.9]{1_y_stepd_10_3_1_0.png}
    \caption{Выход системы при $d_2(t)=r_2(t),r(t)=1$}
    \label{fig:1_y_stepd_10_3_1_0}
\end{figure}
\begin{figure}[H]
    \centering
    \includegraphics[scale=0.9]{1_u_stepd_10_3_1_0.png}
    \caption{Дискретный ПИД регулятор при $d_2(t)=r_2(t),r(t)=1$}
    \label{fig:1_u_stepd_10_3_1_0}
\end{figure}
\begin{figure}[H]
    \centering
    \includegraphics[scale=0.9]{1_e_stepd_10_3_1_0.png}
    \caption{Ошибка $e=r_2(t)-y(t),r(t)=1$}
    \label{fig:1_e_stepd_10_3_1_0}
\end{figure}


\begin{figure}[H]
    \centering
    \includegraphics[scale=0.9]{1_y_stepd_10_m1_1_0.png}
    \caption{Выход системы при $d_3(t)=r_3(t),r(t)=1$}
    \label{fig:1_y_stepd_10_m1_1_0}
\end{figure}
\begin{figure}[H]
    \centering
    \includegraphics[scale=0.9]{1_u_stepd_10_m1_1_0.png}
    \caption{Дискретный ПИД регулятор при $d_3(t)=r_3(t),r(t)=1$}
    \label{fig:1_u_stepd_10_m1_1_0}
\end{figure}
\begin{figure}[H]
    \centering
    \includegraphics[scale=0.9]{1_e_stepd_10_m1_1_0.png}
    \caption{Ошибка $e=r_3(t)-y(t),r(t)=1$}
    \label{fig:1_e_stepd_10_m1_1_0}
\end{figure}


Оставим $r(t)=1$ (см. рис. (\ref{fig:1_r_1})).


Исследуем возмущающее воздействие, изменяющеся по случайному закону
(гауссовский шум):
\begin{figure}[H]
    \centering
    \includegraphics[scale=0.9]{1_d_0.0001.png}
    \caption{Возмущающее воздействие $d_{g,1}(t)$}
    \label{fig:1_d_g_00001}
\end{figure}
\begin{figure}[H]
    \centering
    \includegraphics[scale=0.9]{1_y_g_0.0001.png}
    \caption{Выход системы при $d_{g,1}(t),r(t)=1$}
    \label{fig:1_y_g_00001}
\end{figure}
\begin{figure}[H]
    \centering
    \includegraphics[scale=0.9]{1_u_g_0.0001.png}
    \caption{Дискретный ПИД регулятор при $d_{g,1}(t),r(t)=1$}
    \label{fig:1_u_g_00001}
\end{figure}
\begin{figure}[H]
    \centering
    \includegraphics[scale=0.9]{1_e_g_0.0001.png}
    \caption{Ошибка $e=d_{g,1}(t)-y(t),r(t)=1$}
    \label{fig:1_e_g_00001}
\end{figure}


\begin{figure}[H]
    \centering
    \includegraphics[scale=0.9]{1_d_0.01.png}
    \caption{Возмущающее воздействие $d_{g,2}(t)$}
    \label{fig:1_d_g_001}
\end{figure}
\begin{figure}[H]
    \centering
    \includegraphics[scale=0.9]{1_y_g_0.01.png}
    \caption{Выход системы при $d_{g,2}(t),r(t)=1$}
    \label{fig:1_y_g_001}
\end{figure}
\begin{figure}[H]
    \centering
    \includegraphics[scale=0.9]{1_u_g_0.01.png}
    \caption{Дискретный ПИД регулятор при $d_{g,2}(t),r(t)=1$}
    \label{fig:1_u_g_001}
\end{figure}
\begin{figure}[H]
    \centering
    \includegraphics[scale=0.9]{1_e_g_0.01.png}
    \caption{Ошибка $e=d_{g,2}(t)-y(t),r(t)=1$}
    \label{fig:1_e_g_001}
\end{figure}


Тут будет вывод.


\subsection{Период дискретизации и качество процесса управления}
Установим значение периода дискретизации
в модели экстраполятора нулевого порядка $T=T_1/4\approx0.338$.


Рассчитаем и установим значения параметров
дискретного регулятора:
$$
z_1=d_1=e^{-\frac{T}{T_1}}\approx0.779,
$$
$$
z_2=d_2=e^{-\frac{T}{T_2}}\approx0.755,
$$
$$
\text{num}(z)=z^2+\left( -d_1-d_2 \right)z+d_1d_2=z^2-1.534z+0.588
$$
\end{document}