\documentclass[a4paper,14pt]{extarticle}

\usepackage[T2A]{fontenc}
\usepackage[utf8]{inputenc}
\usepackage[english, russian]{babel}

\usepackage[left=30mm, right=10mm, top=20mm, bottom=20mm]{geometry}

\usepackage{tempora}
\usepackage{setspace}
\onehalfspacing

\usepackage{titlesec}
\titleformat{\section}[block]{\bfseries\centering\MakeUppercase}{\thesection.}{1em}{}
\titleformat{\subsection}[block]{\bfseries}{\thesubsection.}{1em}{}
\titleformat{\subsubsection}[block]{\bfseries}{\thesubsubsection.}{1em}{}

\renewcommand{\contentsname}{\hfill \textbf{СОДЕРЖАНИЕ} \hfill\null}

\usepackage{indentfirst}
\setlength{\parindent}{1.25cm}

\usepackage{amsmath, amsfonts, amssymb}
\usepackage{graphicx}
\usepackage{caption}
\usepackage{subcaption}
\usepackage{float}
\usepackage{tikz}
\usetikzlibrary{patterns}
\usepackage{cmap}
\usepackage{hyperref}
\usepackage{xcolor}
\usepackage{listings}

\definecolor{LightGray}{gray}{0.7}

\lstdefinestyle{code}{
    language=Python, % change if needed
    basicstyle=\small\ttfamily,
    numbers=left,
    numberstyle=\small\color{LightGray},
    stepnumber=1,
    numbersep=5pt,
    backgroundcolor=\color{white},
    showspaces=false,
    showstringspaces=false,
    showtabs=false,
    tabsize=4,
    captionpos=b,
    breaklines=true,
    breakatwhitespace=false,
    frame=single,
    rulecolor=\color{LightGray},
    linewidth=\linewidth,
    keywordstyle=\color{blue}\bfseries,
    commentstyle=\color{green!40!black},
    stringstyle=\color{violet},
    escapeinside={\%*}{*)},
    xleftmargin=10pt,
    xrightmargin=10pt,
    framexleftmargin=0pt,
    framexrightmargin=0pt
}
\lstset{style=code}

\hypersetup{
    colorlinks=true,
    linkcolor=blue,
    filecolor=magenta,
    urlcolor=cyan,
    pdftitle={ncs1},
    pdfauthor={Rumyantsev Alexey},
    pdfsubject={control},
    pdfkeywords={LaTeX, PDF},
    pdfpagemode=FullScreen,
}

\graphicspath{{src/images/}}

\begin{document}

\begin{titlepage}
    \begin{center}
        МИНИСТЕРСТВО НАУКИ И ВЫСШЕГО ОБРАЗОВАНИЯ РОССИЙСКОЙ ФЕДЕРАЦИИ\\
        \vspace*{2.5mm}
        Федеральное государственное автономное образовательное учреждение высшего образования
        «НАЦИОНАЛЬНЫЙ ИССЛЕДОВАТЕЛЬСКИЙ УНИВЕРСИТЕТ ИТМО»\\
        \vspace*{2.5mm}
        \textbf{ФАКУЛЬТЕТ СИСТЕМ УПРАВЛЕНИЯ И РОБОТОТЕХНИКИ}
        \vfill

        {\large ОТЧЕТ ПО ЛАБОРАТОРНОЙ РАБОТЕ №3}\\
        {\large по дисциплине}\\
        {\large\bfseries «ДИСКРЕТНЫЕ СИСТЕМЫ УПРАВЛЕНИЯ»}\\
        {\large на тему}\\
        {\large\bfseries «ДИСКРЕТНЫЙ ПИД РЕГУЛЯТОР»}\\
        Вариант 20
        \vfill

        \begin{flushright}
            Выполнил: студент гр. R3441\\
            Румянцев А. А.\medskip\\

            Проверил: преподаватель\\
            Краснов А. Ю.
        \end{flushright}
        \vfill

        Санкт-Петербург\\
        2025
    \end{center}
\end{titlepage}

\setcounter{page}{2}
\tableofcontents
\newpage

\section{Исходные данные}
Исходные данные варианта 20:
\begin{table}[h!]
\centering
\begin{tabular}{|c|c|}
\hline
$T_1$ & $T_2$\\ \hline
1.35 & 1.2 \\ \hline
\end{tabular}
\end{table}


$T_1, T_2$ -- постоянные времени ОУ.


\section{Выполнение работы}
\subsection{Модель системы}
Модель системы в Simulink:
\begin{figure}[H]
    \centering
    \includegraphics[scale=0.25]{sch1.png}
    \caption{Схема моделирования цифровой САУ температуры}
    \label{fig:sch1}
\end{figure}


\subsection{Значения параметров схемы}
Установим значение периода дискретизации
в модели экстраполятора нулевого порядка
$T=T_1/2\approx0.675$.


Рассчитаем значения полюсов
приведенной непрерывной части:
$$
z_1=d_1=e^{-\frac{T}{T_1}}\approx0.607,
$$
$$
z_2=d_2=e^{-\frac{T}{T_2}}\approx0.570
$$


Посчитаем значения коэффициентов полинома
дискретного регулятора:
$$
\text{num}(z)=z^2+\left( -d_1-d_2 \right)z+d_1d_2=z^2-1.176z+0.346
$$


Установим полученные коэффициенты
полинома в блок Discrete Transfer Fcn и
значения постоянных времени $T_1,T_2$
в модель ОУ.


\subsection{Значение коэффициента передачи регулятора}
Проверим выход системы при значениях коэффициента передачи регулятора $q_0=0.0007,q_0=0.0008$
при задающем воздействии $r(t)=1, T=T_1/2$:
\begin{figure}[H]
    \centering
    \includegraphics[scale=0.9]{q0_07_y.png}
    \caption{Выход системы при $q_0=0.0007,r(t)=1,T=T_1/2$}
    \label{fig:q0_07_y}
\end{figure}
% \begin{figure}[H]
%     \centering
%     \includegraphics[scale=0.9]{q0_07_u.png}
%     \caption{Дискретный ПИД регулятор при $q_0=0.0007,r(t)=1,T=T_1/2$}
%     \label{fig:q0_07_u}
% \end{figure}
\begin{figure}[H]
    \centering
    \includegraphics[scale=0.9]{q0_08_y.png}
    \caption{Выход системы при $q_0=0.0008,r(t)=1,T=T_1/2$}
    \label{fig:q0_08_y}
\end{figure}
% \begin{figure}[H]
%     \centering
%     \includegraphics[scale=0.9]{q0_08_u.png}
%     \caption{Дискретный ПИД регулятор при $q_0=0.0008,r(t)=1,T=T_1/2$}
%     \label{fig:q0_08_u}
% \end{figure}
\begin{figure}[H]
    \centering
    \includegraphics[scale=0.9]{1_r_1.png}
    \caption{Задающее воздействие $r(t)=1$}
    \label{fig:1_r_1}
\end{figure}


При $q_0=0.0008$ система устойчива и имеет слабоколебательные
переходные процессы, при $q_0<0.0008$ сходится монотонно.


\subsection{Процессы на выходе дискретного регулятора и системы}
Исследуем ступенчатое изменение задающего воздействия при $T=T_1/2$:
\begin{figure}[H]
    \centering
    \includegraphics[scale=0.9]{1_r_step_10_0_1_0.png}
    \caption{Задающее воздействие $r_1(t)$}
    \label{fig:1_r_step_10_0_1_0}
\end{figure}
\begin{figure}[H]
    \centering
    \includegraphics[scale=0.9]{1_y_step_10_0_1_0.png}
    \caption{Выход системы при $r_1(t),T=T_1/2$}
    \label{fig:1_y_step_10_0_1_0}
\end{figure}
\begin{figure}[H]
    \centering
    \includegraphics[scale=0.9]{1_u_step_10_0_1_0.png}
    \caption{Дискретный ПИД регулятор при $r_1(t),T=T_1/2$}
    \label{fig:1_u_step_10_0_1_0}
\end{figure}
\begin{figure}[H]
    \centering
    \includegraphics[scale=0.9]{1_e_step_10_0_1_0.png}
    \caption{Ошибка $e=r_1(t)-y(t),T=T_1/2$}
    \label{fig:1_e_step_10_0_1_0}
\end{figure}


\begin{figure}[H]
    \centering
    \includegraphics[scale=0.9]{1_r_step_10_3_1_0.png}
    \caption{Задающее воздействие $r_2(t)$}
    \label{fig:1_r_step_10_3_1_0}
\end{figure}
\begin{figure}[H]
    \centering
    \includegraphics[scale=0.9]{1_y_step_10_3_1_0.png}
    \caption{Выход системы при $r_2(t),T=T_1/2$}
    \label{fig:1_y_step_10_3_1_0}
\end{figure}
\begin{figure}[H]
    \centering
    \includegraphics[scale=0.9]{1_u_step_10_3_1_0.png}
    \caption{Дискретный ПИД регулятор при $r_2(t),T=T_1/2$}
    \label{fig:1_u_step_10_3_1_0}
\end{figure}
\begin{figure}[H]
    \centering
    \includegraphics[scale=0.9]{1_e_step_10_3_1_0.png}
    \caption{Ошибка $e=r_2(t)-y(t),T=T_1/2$}
    \label{fig:1_e_step_10_3_1_0}
\end{figure}


\begin{figure}[H]
    \centering
    \includegraphics[scale=0.9]{1_r_step_10_m1_1_0.png}
    \caption{Задающее воздействие $r_3(t)$}
    \label{fig:1_r_step_10_m1_1_0}
\end{figure}
\begin{figure}[H]
    \centering
    \includegraphics[scale=0.9]{1_y_step_10_m1_1_0.png}
    \caption{Выход системы при $r_3(t),T=T_1/2$}
    \label{fig:1_y_step_10_m1_1_0}
\end{figure}
\begin{figure}[H]
    \centering
    \includegraphics[scale=0.9]{1_u_step_10_m1_1_0.png}
    \caption{Дискретный ПИД регулятор при $r_3(t),T=T_1/2$}
    \label{fig:1_u_step_10_m1_1_0}
\end{figure}
\begin{figure}[H]
    \centering
    \includegraphics[scale=0.9]{1_e_step_10_m1_1_0.png}
    \caption{Ошибка $e=r_3(t)-y(t),T=T_1/2$}
    \label{fig:1_e_step_10_m1_1_0}
\end{figure}


Положим $r(t)=1$ (см. рис. (\ref{fig:1_r_1})).


Исследуем ступенчатое изменение возмущающего воздействия $d(t)$
полагая, что $d_i(t)=r_i(t)$ (см. рис. (\ref{fig:1_r_step_10_0_1_0}),
(\ref{fig:1_r_step_10_3_1_0}), (\ref{fig:1_r_step_10_m1_1_0})) при $T=T_1/2$:
\begin{figure}[H]
    \centering
    \includegraphics[scale=0.9]{1_y_stepd_10_0_1_0.png}
    \caption{Выход системы при $d_1(t)=r_1(t),r(t)=1,T=T_1/2$}
    \label{fig:1_y_stepd_10_0_1_0}
\end{figure}
\begin{figure}[H]
    \centering
    \includegraphics[scale=0.9]{1_u_stepd_10_0_1_0.png}
    \caption{Дискретный ПИД регулятор при $d_1(t)=r_1(t),r(t)=1,T=T_1/2$}
    \label{fig:1_u_stepd_10_0_1_0}
\end{figure}
\begin{figure}[H]
    \centering
    \includegraphics[scale=0.9]{1_e_stepd_10_0_1_0.png}
    \caption{Ошибка $e=r_1(t)-y(t),r(t)=1,T=T_1/2$}
    \label{fig:1_e_stepd_10_0_1_0}
\end{figure}


\begin{figure}[H]
    \centering
    \includegraphics[scale=0.9]{1_y_stepd_10_3_1_0.png}
    \caption{Выход системы при $d_2(t)=r_2(t),r(t)=1,T=T_1/2$}
    \label{fig:1_y_stepd_10_3_1_0}
\end{figure}
\begin{figure}[H]
    \centering
    \includegraphics[scale=0.9]{1_u_stepd_10_3_1_0.png}
    \caption{Дискретный ПИД регулятор при $d_2(t)=r_2(t),r(t)=1,T=T_1/2$}
    \label{fig:1_u_stepd_10_3_1_0}
\end{figure}
\begin{figure}[H]
    \centering
    \includegraphics[scale=0.9]{1_e_stepd_10_3_1_0.png}
    \caption{Ошибка $e=r_2(t)-y(t),r(t)=1,T=T_1/2$}
    \label{fig:1_e_stepd_10_3_1_0}
\end{figure}


\begin{figure}[H]
    \centering
    \includegraphics[scale=0.9]{1_y_stepd_10_m1_1_0.png}
    \caption{Выход системы при $d_3(t)=r_3(t),r(t)=1,T=T_1/2$}
    \label{fig:1_y_stepd_10_m1_1_0}
\end{figure}
\begin{figure}[H]
    \centering
    \includegraphics[scale=0.9]{1_u_stepd_10_m1_1_0.png}
    \caption{Дискретный ПИД регулятор при $d_3(t)=r_3(t),r(t)=1,T=T_1/2$}
    \label{fig:1_u_stepd_10_m1_1_0}
\end{figure}
\begin{figure}[H]
    \centering
    \includegraphics[scale=0.9]{1_e_stepd_10_m1_1_0.png}
    \caption{Ошибка $e=r_3(t)-y(t),r(t)=1,T=T_1/2$}
    \label{fig:1_e_stepd_10_m1_1_0}
\end{figure}


Оставим $r(t)=1$ (см. рис. (\ref{fig:1_r_1})).


Исследуем возмущающее воздействие, изменяющеся по случайному закону
(гауссовский шум) при $T=T_1/2$:
\begin{figure}[H]
    \centering
    \includegraphics[scale=0.9]{1_d_0.0001.png}
    \caption{Возмущающее воздействие $d_{g,1}(t)$}
    \label{fig:1_d_g_00001}
\end{figure}
\begin{figure}[H]
    \centering
    \includegraphics[scale=0.9]{1_y_g_0.0001.png}
    \caption{Выход системы при $d_{g,1}(t),r(t)=1,T=T_1/2$}
    \label{fig:1_y_g_00001}
\end{figure}
\begin{figure}[H]
    \centering
    \includegraphics[scale=0.9]{1_u_g_0.0001.png}
    \caption{Дискретный ПИД регулятор при $d_{g,1}(t),r(t)=1,T=T_1/2$}
    \label{fig:1_u_g_00001}
\end{figure}
\begin{figure}[H]
    \centering
    \includegraphics[scale=0.9]{1_e_g_0.0001.png}
    \caption{Ошибка $e=d_{g,1}(t)-y(t),r(t)=1,T=T_1/2$}
    \label{fig:1_e_g_00001}
\end{figure}


\begin{figure}[H]
    \centering
    \includegraphics[scale=0.9]{1_d_0.01.png}
    \caption{Возмущающее воздействие $d_{g,2}(t)$}
    \label{fig:1_d_g_001}
\end{figure}
\begin{figure}[H]
    \centering
    \includegraphics[scale=0.9]{1_y_g_0.01.png}
    \caption{Выход системы при $d_{g,2}(t),r(t)=1,T=T_1/2$}
    \label{fig:1_y_g_001}
\end{figure}
\begin{figure}[H]
    \centering
    \includegraphics[scale=0.9]{1_u_g_0.01.png}
    \caption{Дискретный ПИД регулятор при $d_{g,2}(t),r(t)=1,T=T_1/2$}
    \label{fig:1_u_g_001}
\end{figure}
\begin{figure}[H]
    \centering
    \includegraphics[scale=0.9]{1_e_g_0.01.png}
    \caption{Ошибка $e=d_{g,2}(t)-y(t),r(t)=1,T=T_1/2$}
    \label{fig:1_e_g_001}
\end{figure}


Вне зависимости от ступенчатого входящего воздействия
или возмущения система около положения равновесия
имеет слабоколебательные процессы. Время переходного
процесса околоидентично.


При воздействии маленького случайного шума
система колеблется около положения равновесия.
При увеличении амплитуды такого внешнего воздействия
колебания усиливаются. Такое поведение ожидаемо, так
как случайный шум нельзя скомпенсировать полностью.


В случаях, когда начальный задающий сигнал дальше от конечного,
регулятору требуется больше энергии на стабилизацию системы.


\subsection{Период дискретизации и качество процесса управления}
Установим значение периода дискретизации
в модели экстраполятора нулевого порядка $T=T_1/4\approx0.338$.


Рассчитаем и установим значения параметров
дискретного регулятора:
$$
z_1=d_1=e^{-\frac{T}{T_1}}\approx0.779,
$$
$$
z_2=d_2=e^{-\frac{T}{T_2}}\approx0.755,
$$
$$
\text{num}(z)=z^2+\left( -d_1-d_2 \right)z+d_1d_2=z^2-1.534z+0.588
$$


Положим $q_0=0.0008$, как и ранее.


Исследуем ступенчатое изменение задающего воздействия (см. рис. (\ref{fig:1_r_step_10_0_1_0}), (\ref{fig:1_r_step_10_3_1_0}), (\ref{fig:1_r_step_10_m1_1_0})) при $T=T_1/4$:
\begin{figure}[H]
    \centering
    \includegraphics[scale=0.9]{2_y_step_10_0_1_0.png}
    \caption{Выход системы при $r_1(t),T=T_1/4$}
    \label{fig:2_y_step_10_0_1_0}
\end{figure}
\begin{figure}[H]
    \centering
    \includegraphics[scale=0.9]{2_u_step_10_0_1_0.png}
    \caption{Дискретный ПИД регулятор при $r_1(t),T=T_1/4$}
    \label{fig:2_u_step_10_0_1_0}
\end{figure}
\begin{figure}[H]
    \centering
    \includegraphics[scale=0.9]{2_e_step_10_0_1_0.png}
    \caption{Ошибка $e=r_1(t)-y(t),T=T_1/4$}
    \label{fig:2_e_step_10_0_1_0}
\end{figure}


\begin{figure}[H]
    \centering
    \includegraphics[scale=0.9]{2_y_step_10_3_1_0.png}
    \caption{Выход системы при $r_2(t),T=T_1/4$}
    \label{fig:2_y_step_10_3_1_0}
\end{figure}
\begin{figure}[H]
    \centering
    \includegraphics[scale=0.9]{2_u_step_10_3_1_0.png}
    \caption{Дискретный ПИД регулятор при $r_2(t),T=T_1/4$}
    \label{fig:2_u_step_10_3_1_0}
\end{figure}
\begin{figure}[H]
    \centering
    \includegraphics[scale=0.9]{2_e_step_10_3_1_0.png}
    \caption{Ошибка $e=r_2(t)-y(t),T=T_1/4$}
    \label{fig:2_e_step_10_3_1_0}
\end{figure}


\begin{figure}[H]
    \centering
    \includegraphics[scale=0.9]{2_y_step_10_m1_1_0.png}
    \caption{Выход системы при $r_3(t),T=T_1/4$}
    \label{fig:2_y_step_10_m1_1_0}
\end{figure}
\begin{figure}[H]
    \centering
    \includegraphics[scale=0.9]{2_u_step_10_m1_1_0.png}
    \caption{Дискретный ПИД регулятор при $r_3(t),T=T_1/4$}
    \label{fig:2_u_step_10_m1_1_0}
\end{figure}
\begin{figure}[H]
    \centering
    \includegraphics[scale=0.9]{2_e_step_10_m1_1_0.png}
    \caption{Ошибка $e=r_3(t)-y(t),T=T_1/4$}
    \label{fig:2_e_step_10_m1_1_0}
\end{figure}


Во всех случаях при уменьшении
пероида дискретизации в системе
стало больше осцилляций,
время переходного процесса увеличилось,
управления стало затрачиваться больше.


\subsection{Неточность компенсации полюсов объекта управления и качество процесса управления}
Увеличим значение постоянной времени $T_2$ ОУ на 20\%: $T_{2_{+20\%}}=1.2\cdot T_2=1.44$.


Примем $T=T_1/4$.


Рассчитаем параметры дискретного регулятора и установим их в модель:
$$
z_1=d_1=e^{-\frac{T}{T_1}}\approx0.779,
$$
$$
z_2=d_2=e^{-\frac{T}{T_{2_{+20\%}}}}\approx0.791,
$$
$$
\text{num}(z)=z^2+\left( -d_1-d_2 \right)z+d_1d_2=z^2-1.570z+0.616
$$


Положим $q_0=0.0008$, как и ранее.


Исследуем ступенчатое изменение задающего воздействия (см. рис. (\ref{fig:1_r_step_10_0_1_0}), (\ref{fig:1_r_step_10_3_1_0}), (\ref{fig:1_r_step_10_m1_1_0})) при $T=T_1/4$:
\begin{figure}[H]
    \centering
    \includegraphics[scale=0.9]{3_y_step_10_0_1_0.png}
    \caption{Выход системы при $r_1(t),T=T_1/4,T_2=1.44$}
    \label{fig:3_y_step_10_0_1_0}
\end{figure}
\begin{figure}[H]
    \centering
    \includegraphics[scale=0.9]{3_u_step_10_0_1_0.png}
    \caption{Дискретный ПИД регулятор при $r_1(t),T=T_1/4,T_2=1.44$}
    \label{fig:3_u_step_10_0_1_0}
\end{figure}
\begin{figure}[H]
    \centering
    \includegraphics[scale=0.9]{3_e_step_10_0_1_0.png}
    \caption{Ошибка $e=r_1(t)-y(t),T=T_1/4,T_2=1.44$}
    \label{fig:3_e_step_10_0_1_0}
\end{figure}


\begin{figure}[H]
    \centering
    \includegraphics[scale=0.9]{3_y_step_10_3_1_0.png}
    \caption{Выход системы при $r_2(t),T=T_1/4,T_2=1.44$}
    \label{fig:3_y_step_10_3_1_0}
\end{figure}
\begin{figure}[H]
    \centering
    \includegraphics[scale=0.9]{3_u_step_10_3_1_0.png}
    \caption{Дискретный ПИД регулятор при $r_2(t),T=T_1/4,T_2=1.44$}
    \label{fig:3_u_step_10_3_1_0}
\end{figure}
\begin{figure}[H]
    \centering
    \includegraphics[scale=0.9]{3_e_step_10_3_1_0.png}
    \caption{Ошибка $e=r_2(t)-y(t),T=T_1/4,T_2=1.44$}
    \label{fig:3_e_step_10_3_1_0}
\end{figure}


\begin{figure}[H]
    \centering
    \includegraphics[scale=0.9]{3_y_step_10_m1_1_0.png}
    \caption{Выход системы при $r_3(t),T=T_1/4,T_2=1.44$}
    \label{fig:3_y_step_10_m1_1_0}
\end{figure}
\begin{figure}[H]
    \centering
    \includegraphics[scale=0.9]{3_u_step_10_m1_1_0.png}
    \caption{Дискретный ПИД регулятор при $r_3(t),T=T_1/4,T_2=1.44$}
    \label{fig:3_u_step_10_m1_1_0}
\end{figure}
\begin{figure}[H]
    \centering
    \includegraphics[scale=0.9]{3_e_step_10_m1_1_0.png}
    \caption{Ошибка $e=r_3(t)-y(t),T=T_1/4,T_2=1.44$}
    \label{fig:3_e_step_10_m1_1_0}
\end{figure}


В системе стало больше осцилляций,
время переходного процесса увеличилось,
управления затрачивается больше.


По сравнению со случаями для исследования
периода дискретизации амплитуды колебаний
в этом случае несколько меньше.


Уменьшим значение постоянной времени $T_2$
на 20\%: $T_{2_{-20\%}}=0.8\cdot T_2=0.96$.


Примем $T=T_1/4$.


Рассчитаем параметры дискретного регулятора и установим их в модель:
$$
z_1=d_1=e^{-\frac{T}{T_1}}\approx0.779,
$$
$$
z_2=d_2=e^{-\frac{T}{T_{2_{-20\%}}}}\approx0.704,
$$
$$
\text{num}(z)=z^2+\left( -d_1-d_2 \right)z+d_1d_2=z^2-1.482z+0.548
$$


Положим $q_0=0.0008$, как и ранее.


Исследуем ступенчатое изменение задающего воздействия (см. рис. (\ref{fig:1_r_step_10_0_1_0}), (\ref{fig:1_r_step_10_3_1_0}), (\ref{fig:1_r_step_10_m1_1_0})) при $T=T_1/4$:
\begin{figure}[H]
    \centering
    \includegraphics[scale=0.9]{4_y_step_10_0_1_0.png}
    \caption{Выход системы при $r_1(t),T=T_1/4,T_2=0.96$}
    \label{fig:4_y_step_10_0_1_0}
\end{figure}
\begin{figure}[H]
    \centering
    \includegraphics[scale=0.9]{4_u_step_10_0_1_0.png}
    \caption{Дискретный ПИД регулятор при $r_1(t),T=T_1/4,T_2=0.96$}
    \label{fig:4_u_step_10_0_1_0}
\end{figure}
\begin{figure}[H]
    \centering
    \includegraphics[scale=0.9]{4_e_step_10_0_1_0.png}
    \caption{Ошибка $e=r_1(t)-y(t),T=T_1/4,T_2=0.96$}
    \label{fig:4_e_step_10_0_1_0}
\end{figure}


\begin{figure}[H]
    \centering
    \includegraphics[scale=0.9]{4_y_step_10_3_1_0.png}
    \caption{Выход системы при $r_2(t),T=T_1/4,T_2=0.96$}
    \label{fig:4_y_step_10_3_1_0}
\end{figure}
\begin{figure}[H]
    \centering
    \includegraphics[scale=0.9]{4_u_step_10_3_1_0.png}
    \caption{Дискретный ПИД регулятор при $r_2(t),T=T_1/4,T_2=0.96$}
    \label{fig:4_u_step_10_3_1_0}
\end{figure}
\begin{figure}[H]
    \centering
    \includegraphics[scale=0.9]{4_e_step_10_3_1_0.png}
    \caption{Ошибка $e=r_2(t)-y(t),T=T_1/4,T_2=0.96$}
    \label{fig:4_e_step_10_3_1_0}
\end{figure}


\begin{figure}[H]
    \centering
    \includegraphics[scale=0.9]{4_y_step_10_m1_1_0.png}
    \caption{Выход системы при $r_3(t),T=T_1/4,T_2=0.96$}
    \label{fig:4_y_step_10_m1_1_0}
\end{figure}
\begin{figure}[H]
    \centering
    \includegraphics[scale=0.9]{4_u_step_10_m1_1_0.png}
    \caption{Дискретный ПИД регулятор при $r_3(t),T=T_1/4,T_2=0.96$}
    \label{fig:4_u_step_10_m1_1_0}
\end{figure}
\begin{figure}[H]
    \centering
    \includegraphics[scale=0.9]{4_e_step_10_m1_1_0.png}
    \caption{Ошибка $e=r_3(t)-y(t),T=T_1/4,T_2=0.96$}
    \label{fig:4_e_step_10_m1_1_0}
\end{figure}


В системе появились осцилляции,
время переходного процесса увеличилось,
управления затрачивается больше.


Амплитуды колебаний в этом случае
несколько больше, чем в исследованиях
периода дискретизации и $T_{2_{+20\%}}$.


\section{Вывод}
В ходе выполнения лабораторной работы
была исследована модель цифровой САУ
температуры. Было подобрано значение
коэффициента передачи регулятора
для устойчивого слабоколебательного
процесса. Было исследовано
влияние различных ступенчатых
задающих воздействий и возмущений,
а также случайного шума на поведение
системы и регулятора. Был сделан вывод,
что такие воздействия почти не
влияют на слабоколебательность процесса
и время переходного процесса, кроме 
случайного шума -- в его случае
система колеблется около положения равновесия,
так как случайный шум нельзя полностью скомпенсировать.
Было исследование влияние на модель периода дискретизации
и неточности компенсации полюсов ОУ. Результаты моделирования
показали, что в этих случаях в системе становится
больше осцилляций, время переходного процесса увеличивается,
управления затрачивается больше.
\end{document}