\documentclass[a4paper,14pt]{extarticle}

\usepackage[T2A]{fontenc}
\usepackage[utf8]{inputenc}
\usepackage[english, russian]{babel}

\usepackage[left=30mm, right=10mm, top=20mm, bottom=20mm]{geometry}

\usepackage{tempora}
\usepackage{setspace}
\onehalfspacing

\usepackage{titlesec}
\titleformat{\section}[block]{\bfseries\centering\MakeUppercase}{\thesection.}{1em}{}
\titleformat{\subsection}[block]{\bfseries}{\thesubsection.}{1em}{}
\titleformat{\subsubsection}[block]{\bfseries}{\thesubsubsection.}{1em}{}

\renewcommand{\contentsname}{\hfill \textbf{СОДЕРЖАНИЕ} \hfill\null}

\usepackage{indentfirst}
\setlength{\parindent}{1.25cm}

\usepackage{amsmath, amsfonts, amssymb}
\usepackage{graphicx}
\usepackage{caption}
\usepackage{subcaption}
\usepackage{float}
\usepackage{tikz}
\usetikzlibrary{patterns}
\usepackage{cmap}
\usepackage{hyperref}
\usepackage{xcolor}
\usepackage{listings}

\definecolor{LightGray}{gray}{0.7}

\lstdefinestyle{code}{
    language=Python, % change if needed
    basicstyle=\small\ttfamily,
    numbers=left,
    numberstyle=\small\color{LightGray},
    stepnumber=1,
    numbersep=5pt,
    backgroundcolor=\color{white},
    showspaces=false,
    showstringspaces=false,
    showtabs=false,
    tabsize=4,
    captionpos=b,
    breaklines=true,
    breakatwhitespace=false,
    frame=single,
    rulecolor=\color{LightGray},
    linewidth=\linewidth,
    keywordstyle=\color{blue}\bfseries,
    commentstyle=\color{green!40!black},
    stringstyle=\color{violet},
    escapeinside={\%*}{*)},
    xleftmargin=10pt,
    xrightmargin=10pt,
    framexleftmargin=0pt,
    framexrightmargin=0pt
}
\lstset{style=code}

\hypersetup{
    colorlinks=true,
    linkcolor=blue,
    filecolor=magenta,
    urlcolor=cyan,
    pdftitle={dcsl3},
    pdfauthor={Rumyantsev Alexey},
    pdfsubject={control},
    pdfkeywords={LaTeX, PDF},
    pdfpagemode=FullScreen,
}

\graphicspath{{src/images/}}

\begin{document}

\begin{titlepage}
    \begin{center}
        МИНИСТЕРСТВО НАУКИ И ВЫСШЕГО ОБРАЗОВАНИЯ РОССИЙСКОЙ ФЕДЕРАЦИИ\\
        \vspace*{2.5mm}
        Федеральное государственное автономное образовательное учреждение высшего образования
        «НАЦИОНАЛЬНЫЙ ИССЛЕДОВАТЕЛЬСКИЙ УНИВЕРСИТЕТ ИТМО»\\
        \vspace*{2.5mm}
        \textbf{ФАКУЛЬТЕТ СИСТЕМ УПРАВЛЕНИЯ И РОБОТОТЕХНИКИ}
        \vfill

        {\large ОТЧЕТ ПО ЛАБОРАТОРНОЙ РАБОТЕ №3}\\
        {\large по дисциплине}\\
        {\large\bfseries «ДИСКРЕТНЫЕ СИСТЕМЫ УПРАВЛЕНИЯ»}\\
        {\large на тему}\\
        {\large\bfseries «ДИСКРЕТНЫЙ ПИД РЕГУЛЯТОР»}\\
        Вариант 20
        \vfill

        \begin{flushright}
            Выполнил: студент гр. R3441\\
            Румянцев А. А.\medskip\\

            Проверил: преподаватель\\
            Краснов А. Ю.
        \end{flushright}
        \vfill

        Санкт-Петербург\\
        2025
    \end{center}
\end{titlepage}

\setcounter{page}{2}
\tableofcontents
\newpage

\section{Исходные данные}
Исходные данные варианта 20:
\begin{table}[h!]
\centering
\begin{tabular}{|c|c|}
\hline
$T_1$ & $T_2$\\ \hline
1.35 & 1.2 \\ \hline
\end{tabular}
\end{table}


$T_1, T_2$ -- постоянные времени ОУ.


\section{Выполнение работы}
\subsection{Модель системы}
Модель системы в Simulink:
\begin{figure}[H]
    \centering
    \includegraphics[scale=0.25]{sch1.png}
    \caption{Схема моделирования цифровой САУ температуры}
    \label{fig:sch1}
\end{figure}


\subsection{Значения параметров схемы}
Установим значение периода дискретизации
в модели экстраполятора нулевого порядка
$T=T_1/2\approx0.675$.


Рассчитаем значения полюсов
приведенной непрерывной части:
$$
z_1=d_1=e^{-\frac{T}{T_1}}\approx0.607,
$$
$$
z_2=d_2=e^{-\frac{T}{T_2}}\approx0.570
$$


Посчитаем значения коэффициентов полинома
дискретного регулятора:
$$
\text{num}(z)=z^2+\left( -d_1-d_2 \right)z+d_1d_2=z^2-1.176z+0.346
$$


Установим полученные коэффициенты
полинома в блок Discrete Transfer Fcn и
значения постоянных времени $T_1,T_2$
в модель ОУ.


\subsection{Значение коэффициента передачи регулятора}
Задающее/возмущающее ступенчатое воздействие:
\begin{figure}[H]
    \centering
    \includegraphics[scale=0.7]{rd.png}
    \caption{Задающее/возмущающее воздействие $r(t)=d(t)=1$}
    \label{fig:rd}
\end{figure}
Значение $q_0=0.001$, при котором система имеет слабоколебательные переходные процессы:
\begin{figure}[H]
    \centering
    \includegraphics[scale=0.7]{y_q.png}
    \caption{Выход системы при $q_0=0.001,r(t)=1,T=T_1/2$}
    \label{fig:y_q}
\end{figure}


\subsection{Процессы на выходе дискретного регулятора и системы}
Исследуем ступенчатое изменение задающего воздействия при $T=T_1/2$:
\begin{figure}[H]
    \centering
    \includegraphics[scale=0.7]{y_r_t12.png}
    \caption{Выход системы при $r(t)=1,T=T_1/2$}
    \label{fig:y_r_t12}
\end{figure}
\begin{figure}[H]
    \centering
    \includegraphics[scale=0.7]{u_r_t12.png}
    \caption{Выход дискретного ПИД регулятора при $r(t)=1,T=T_1/2$}
    \label{fig:u_r_t12}
\end{figure}
\begin{figure}[H]
    \centering
    \includegraphics[scale=0.7]{e_r_t12.png}
    \caption{Ошибка $e=r(t)-y(t),T=T_1/2$}
    \label{fig:e_r_t12}
\end{figure}


Исследуем ступенчатое изменение возмущающего воздействия:
\begin{figure}[H]
    \centering
    \includegraphics[scale=0.7]{y_rd_t12.png}
    \caption{Выход системы при $d(t)=1,r(t)=1,T=T_1/2$}
    \label{fig:y_rd_t12}
\end{figure}
\begin{figure}[H]
    \centering
    \includegraphics[scale=0.7]{u_rd_t12.png}
    \caption{Выход дискретного ПИД регулятора при $d(t)=1,r(t)=1,T=T_1/2$}
    \label{fig:u_rd_t12}
\end{figure}
\begin{figure}[H]
    \centering
    \includegraphics[scale=0.7]{e_rd_t12.png}
    \caption{Ошибка $e=r(t)-y(t),r(t)=1,d(t)=1,T=T_1/2$}
    \label{fig:e_rd_t12}
\end{figure}


Исследуем возмущающее воздействие, изменяющеся по случайному закону
(гауссовский шум) при $T=T_1/2$:
\begin{figure}[H]
    \centering
    \includegraphics[scale=0.7]{d_g.png}
    \caption{Возмущающее воздействие $d_{g}(t)$}
    \label{fig:d_g}
\end{figure}
\begin{figure}[H]
    \centering
    \includegraphics[scale=0.7]{y_gr_t12.png}
    \caption{Выход системы при $d_{g}(t),r(t)=1,T=T_1/2$}
    \label{fig:y_gr_t12}
\end{figure}
\begin{figure}[H]
    \centering
    \includegraphics[scale=0.7]{u_gr_t12.png}
    \caption{Выход дискретного ПИД регулятора при $d_{g}(t),r(t)=1,T=T_1/2$}
    \label{fig:u_gr_t12}
\end{figure}
\begin{figure}[H]
    \centering
    \includegraphics[scale=0.7]{e_gr_t12.png}
    \caption{Ошибка $e=d_{g}(t)-y(t),r(t)=1,T=T_1/2$}
    \label{fig:e_gr_t12}
\end{figure}


При наличии ступенчатого возмущения время переходного процесса увеличивается,
управления затрачивается больше.


При воздействии небольшого случайного шума
система колеблется около положения равновесия
-- случайный шум нельзя скомпенсировать полностью.


\subsection{Период дискретизации и качество процесса управления}
Установим значение периода дискретизации
в модели экстраполятора нулевого порядка $T=T_1/4\approx0.338$.


Рассчитаем и установим значения параметров
дискретного регулятора:
$$
z_1=d_1=e^{-\frac{T}{T_1}}\approx0.779,
$$
$$
z_2=d_2=e^{-\frac{T}{T_2}}\approx0.755,
$$
$$
\text{num}(z)=z^2+\left( -d_1-d_2 \right)z+d_1d_2=z^2-1.534z+0.588
$$


Исследуем ступенчатое изменение задающего воздействия при $T=T_1/4$:
\begin{figure}[H]
    \centering
    \includegraphics[scale=0.7]{y_r_t14.png}
    \caption{Выход системы при $r(t)=1,T=T_1/4$}
    \label{fig:y_r_t14}
\end{figure}
\begin{figure}[H]
    \centering
    \includegraphics[scale=0.7]{u_r_t14.png}
    \caption{Выход дискретного ПИД регулятора при $r(t)=1,T=T_1/4$}
    \label{fig:u_r_t14}
\end{figure}
\begin{figure}[H]
    \centering
    \includegraphics[scale=0.7]{e_r_t14.png}
    \caption{Ошибка $e=r(t)-y(t),T=T_1/4$}
    \label{fig:e_r_t14}
\end{figure}


При уменьшении
пероида дискретизации в системе
стало больше осцилляций,
время переходного процесса увеличилось,
управления стало затрачиваться больше.


\subsection{Неточность компенсации полюсов объекта управления и качество процесса управления}
Увеличим значение постоянной времени $T_2$ ОУ на 20\%: $T_{2_{+20\%}}=1.2\cdot T_2=1.44$.


Примем $T=T_1/4$.


Рассчитаем параметры дискретного регулятора и установим их в модель:
$$
z_1=d_1=e^{-\frac{T}{T_1}}\approx0.779,
$$
$$
z_2=d_2=e^{-\frac{T}{T_{2_{+20\%}}}}\approx0.791,
$$
$$
\text{num}(z)=z^2+\left( -d_1-d_2 \right)z+d_1d_2=z^2-1.570z+0.616
$$


Исследуем ступенчатое изменение задающего воздействия при $T=T_1/4$:
\begin{figure}[H]
    \centering
    \includegraphics[scale=0.7]{y_r_tp.png}
    \caption{Выход системы при $r(t)=1,T=T_1/4,T_2=1.44$}
    \label{fig:y_r_tp}
\end{figure}
\begin{figure}[H]
    \centering
    \includegraphics[scale=0.7]{u_r_tp.png}
    \caption{Выход дискретного ПИД регулятора при $r(t)=1,T=T_1/4,T_2=1.44$}
    \label{fig:u_r_tp}
\end{figure}
\begin{figure}[H]
    \centering
    \includegraphics[scale=0.7]{e_r_tp.png}
    \caption{Ошибка $e=r(t)-y(t),T=T_1/4,T_2=1.44$}
    \label{fig:e_r_tp}
\end{figure}


В системе стало больше осцилляций,
время переходного процесса увеличилось,
управления затрачивается больше.


По сравнению со случаем исследования
периода дискретизации амплитуды колебаний
несколько меньше.


Уменьшим значение постоянной времени $T_2$
на 20\%: $T_{2_{-20\%}}=0.8\cdot T_2=0.96$.


Рассчитаем параметры дискретного регулятора и установим их в модель:
$$
z_1=d_1=e^{-\frac{T}{T_1}}\approx0.779,
$$
$$
z_2=d_2=e^{-\frac{T}{T_{2_{-20\%}}}}\approx0.704,
$$
$$
\text{num}(z)=z^2+\left( -d_1-d_2 \right)z+d_1d_2=z^2-1.482z+0.548
$$


Исследуем ступенчатое изменение задающего воздействия при $T=T_1/4$:
\begin{figure}[H]
    \centering
    \includegraphics[scale=0.7]{y_r_tm.png}
    \caption{Выход системы при $r(t)=1,T=T_1/4,T_2=0.96$}
    \label{fig:y_r_tm}
\end{figure}
\begin{figure}[H]
    \centering
    \includegraphics[scale=0.7]{u_r_tm.png}
    \caption{Выход дискретного ПИД регулятора при $r(t)=1,T=T_1/4,T_2=0.96$}
    \label{fig:u_r_tm}
\end{figure}
\begin{figure}[H]
    \centering
    \includegraphics[scale=0.7]{e_r_tm.png}
    \caption{Ошибка $e=r(t)-y(t),T=T_1/4,T_2=0.96$}
    \label{fig:e_r_tm}
\end{figure}


В системе появились осцилляции,
время переходного процесса увеличилось,
управления затрачивается больше.


Амплитуды колебаний в этом случае
несколько больше, чем в исследованиях
периода дискретизации и $T_{2_{+20\%}}$.


\section{Вывод}
В ходе выполнения лабораторной работы
была исследована модель цифровой САУ
температуры. Было подобрано значение
коэффициента передачи регулятора
для устойчивого слабоколебательного
процесса. Было исследовано
влияние ступенчатого возмущения и случайного шума на поведение
системы и регулятора. Был сделан вывод,
что в первом случае воздействия почти не
влияют на слабоколебательность процесса, во втором
система колеблется около положения равновесия,
так как случайный шум нельзя полностью скомпенсировать.
Было исследовано влияние на модель периода дискретизации
и неточности компенсации полюсов ОУ. Результаты моделирования
показали, что в этих случаях в системе становится
больше осцилляций, время переходного процесса увеличивается,
управления затрачивается больше.
\end{document}