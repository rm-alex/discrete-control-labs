\documentclass[a4paper,14pt]{extarticle}

\usepackage[T2A]{fontenc}
\usepackage[utf8]{inputenc}
\usepackage[english, russian]{babel}

\usepackage[left=30mm, right=10mm, top=20mm, bottom=20mm]{geometry}

\usepackage{tempora}
\usepackage{setspace}
\onehalfspacing

\usepackage{titlesec}
\titleformat{\section}[block]{\bfseries\centering\MakeUppercase}{\thesection.}{1em}{}
\titleformat{\subsection}[block]{\bfseries}{\thesubsection.}{1em}{}
\titleformat{\subsubsection}[block]{\bfseries}{\thesubsubsection.}{1em}{}

\renewcommand{\contentsname}{\hfill \textbf{СОДЕРЖАНИЕ} \hfill\null}

\usepackage{indentfirst}
\setlength{\parindent}{1.25cm}

\usepackage{amsmath, amsfonts, amssymb}
\usepackage{graphicx}
\usepackage{caption}
\usepackage{subcaption}
\usepackage{float}
\usepackage{tikz}
\usetikzlibrary{patterns}
\usepackage{cmap}
\usepackage{hyperref}
\usepackage{xcolor}
\usepackage{listings}

\definecolor{LightGray}{gray}{0.7}

\lstdefinestyle{code}{
    language=Python, % change if needed
    basicstyle=\small\ttfamily,
    numbers=left,
    numberstyle=\small\color{LightGray},
    stepnumber=1,
    numbersep=5pt,
    backgroundcolor=\color{white},
    showspaces=false,
    showstringspaces=false,
    showtabs=false,
    tabsize=4,
    captionpos=b,
    breaklines=true,
    breakatwhitespace=false,
    frame=single,
    rulecolor=\color{LightGray},
    linewidth=\linewidth,
    keywordstyle=\color{blue}\bfseries,
    commentstyle=\color{green!40!black},
    stringstyle=\color{violet},
    escapeinside={\%*}{*)},
    xleftmargin=10pt,
    xrightmargin=10pt,
    framexleftmargin=0pt,
    framexrightmargin=0pt
}
\lstset{style=code}

\hypersetup{
    colorlinks=true,
    linkcolor=blue,
    filecolor=magenta,
    urlcolor=cyan,
    pdftitle={ncs1},
    pdfauthor={Rumyantsev Alexey},
    pdfsubject={control},
    pdfkeywords={LaTeX, PDF},
    pdfpagemode=FullScreen,
}

\graphicspath{{src/images/}}

\begin{document}

\begin{titlepage}
    \begin{center}
        МИНИСТЕРСТВО НАУКИ И ВЫСШЕГО ОБРАЗОВАНИЯ РОССИЙСКОЙ ФЕДЕРАЦИИ\\
        \vspace*{2.5mm}
        Федеральное государственное автономное образовательное учреждение высшего образования
        «НАЦИОНАЛЬНЫЙ ИССЛЕДОВАТЕЛЬСКИЙ УНИВЕРСИТЕТ ИТМО»\\
        \vspace*{2.5mm}
        \textbf{ФАКУЛЬТЕТ СИСТЕМ УПРАВЛЕНИЯ И РОБОТОТЕХНИКИ}
        \vfill

        {\large ОТЧЕТ ПО ЛАБОРАТОРНОЙ РАБОТЕ №2}\\
        {\large по дисциплине}\\
        {\large\bfseries «ДИСКРЕТНЫЕ СИСТЕМЫ УПРАВЛЕНИЯ»}\\
        {\large на тему}\\
        {\large\bfseries «ДИСКРЕТНЫЕ СТАБИЛИЗИРУЮЩИЕ И СЛЕДЯЩИЕ РЕГУЛЯТОРЫ. НАБЛЮДАТЕЛЬ ДИСКРЕТНОЙ СИСТЕМЫ»}\\
        Вариант 20
        \vfill

        \begin{flushright}
            Выполнил: студент гр. R3441\\
            Румянцев А. А.\medskip\\

            Проверил: преподаватель\\
            Краснов А. Ю.
        \end{flushright}
        \vfill

        Санкт-Петербург\\
        2025
    \end{center}
\end{titlepage}

\setcounter{page}{2}
\tableofcontents
\newpage


\section{Проектирование дискретных стабилизирующих регуляторов}
Исходные данные:
\begin{align*}
    &\text{Тип ОУ: }4,\\
    &k_1=9.71,\\
    &a_0^1=0,\\
    &T_1=1,\\
    &\xi=0,\\
    &k_2=1,\\
    &a_0^2=0,\\
    &T_2=4,\\
    &T=1
\end{align*}


Тип ОУ:
\begin{figure}[H]
    \centering
    \includegraphics[scale=1]{4co.png}
    \caption{Объект управления №4}
    \label{fig:4co}
\end{figure}


В-С-В непрерывного ОУ:
$$
\begin{cases}
    \dot{x}(t)=A_\text{н}x(t)+B_\text{н}u(t),\\
    y(t)=Cx(t)+Du(t)
\end{cases}
$$


Исходная передаточная функция:
$$
W(p)=\frac{k_1}{p^2}
$$


В операторной форме:
$$
W(p)=\frac{Y(p)}{U(p)}=\frac{k_1}{p^2}\Rightarrow Y(p)=\frac{k_1}{p^2}U(p)\Rightarrow p^2Y(p)=k_1U(p)
$$


Заменим $p$ на $d/dt$:
$$
\ddot{y}(t)=k_1u(t)
$$


Перейдем к канонической управляемой форме.


Замены:
$$
x_1=y,\ x_2=\dot{y}
$$


Каноническая управляемая форма:
$$
\begin{cases}
    \dot{x}_1=x_2,\\
    \dot{x}_2=k_1u,\\
    y=x_1
\end{cases}
$$


Матрицы в форме состояния:
$$
A_\text{н}=\begin{bmatrix}
    0&1\\0&0
\end{bmatrix},\ B_\text{н}=\begin{bmatrix}
    0\\k_1
\end{bmatrix},\ C=\begin{bmatrix}
    1&0
\end{bmatrix},\ D=0
$$


Дискретные системы в пространстве состояний
описываются разностными уравнениями:
$$
\begin{cases}
    x(k+1)=Ax(k)+Bu(k),\\
    y(k)=Cx(k)
\end{cases}
$$


Приведем к дискретному виду:
$$
A=\sum\limits_{i=0}^{k}\frac{A^{i}_\text{н}T^{i}}{i!},\ B=\sum\limits_{i=1}^{k}\frac{A^{i-1}_\text{н}T^{i}}{i!}B_\text{н}
$$


Рассмотрим $i=0,1,2$:
$$
A_\text{н}^0=\begin{bmatrix}
    1&0\\0&1
\end{bmatrix},\ A_\text{н}^1=\begin{bmatrix}
    0&1\\0&0
\end{bmatrix},\ A_\text{н}=\begin{bmatrix}
    0&0\\0&0
\end{bmatrix}
$$


Достаточно взять $k=2$:
$$
A=I+A_\text{н}T+\frac{A^2_\text{н}T^2}{2}=\begin{bmatrix}
    1&T\\0&1
\end{bmatrix},
$$
$$
B=A^0_\text{н}TB_\text{н}+\frac{A^1_\text{н}T^2B_\text{н}}{2}=\begin{bmatrix}
    \frac{T^2k_1}{2}\\Tk_1
\end{bmatrix}
$$


Таким образом:
$$
A=\begin{bmatrix}
    1&1\\0&1
\end{bmatrix},\ B=\begin{bmatrix}
    4.855\\9.71
\end{bmatrix},\ C=\begin{bmatrix}
    1&0
\end{bmatrix},\ D=0
$$


Корни характеристического полинома:
$$
\det{\left( z I-A \right)}=0\Rightarrow z_1=1,\ z_2=1
$$


Система неустойчива по Ляпунову, т.к. собственные числа не удовлетворяют $|z_i|<1$.


Матрица управляемости:
$$
U=\begin{bmatrix}
    B &AB
\end{bmatrix}=\begin{bmatrix}
    4.855   &14.565\\
    9.71    &9.71
\end{bmatrix}
$$


Ее ранг:
$$
\operatorname{rank}\left[ U \right]=2
$$


Ранг матрицы управляемости равен размерности состояния $n=2$
-- система полностью управляема.


Матрица наблюдаемости:
$$
V=\begin{bmatrix}
    C\\CA
\end{bmatrix}=\begin{bmatrix}
    1     &0\\
     1     &1
\end{bmatrix}
$$


Ее ранг:
$$
\operatorname{rank}\left[ V \right]=2
$$


Ранг матрицы наблюдаемости равен размерности состояния $n=2$
-- система полностью наблюдаема.


Эталонная модель дискретной системы задается разностными уравнениями:
$$
\begin{cases}
    \xi(k+1)=\Gamma\xi(k),\\
    y(k)=H\xi(k),
\end{cases}
$$
где $\xi(k)$ -- вектор состояния
дискретной эталонной модели,
матрицы $\Gamma,H$ выбираются
в соответствии с требуемыми
показателями качества.


Возьмем оптимальную по быстродействию
дискретную систему -- когда $z_i^*=0,i=\bar{1,n}$:
$$
\Gamma_{n\times n}=\begin{bmatrix}
    z_1&1\\0&z_2
\end{bmatrix}=\begin{bmatrix}
    0&1\\0&0
\end{bmatrix},
$$
$$
H_{l\times n}=\begin{bmatrix}
    1&0
\end{bmatrix}
$$


Пара $\left( H,\Gamma \right)$ полностью наблюдаема по признаку Жордана
-- второй элемент состояния выражается через первый при умножении на $\Gamma$,
который измеряется через $H$.


Эталонный характеристический полином:
$$
D^*(z)=\det{\left( zI-\Gamma \right)}=\begin{vmatrix}
    z&-1\\0&z
\end{vmatrix}=z^2
$$


Оба полюса дискретной эталонной системы находятся в $z=0$ (максимально быстрое затухание).


Уравнение типа Сильвестра:
$$
M\Gamma-AM=-BH
$$


Решение относительно $M$:
$$
M=\begin{bmatrix}
    -4.855  &-14.565\\
    9.71    &9.71
\end{bmatrix}
$$


Матрица линейных стационарных обратных связей:
$$
K=\begin{bmatrix}
    0.103    &0.1545
\end{bmatrix}
$$


Замкнутая система:
$$
x(k+1)=Fx(k),\ F=A-BK,\ u=-Kx(k)
$$

\newpage
Схема моделирования системы:
\begin{figure}[H]
    \centering
    \includegraphics[scale=0.5]{sch1.png}
    \caption{Схема моделирования системы}
    \label{fig:sch1}
\end{figure}


Выполним моделирование замкнутой системы при $y(0)=1,\dot{y}=0$:
\begin{figure}[H]
    \centering
    \includegraphics[scale=0.8]{1x.png}
    \caption{Вектор состояния замкнутой системы}
    \label{fig:1x}
\end{figure}
\begin{figure}[H]
    \centering
    \includegraphics[scale=0.8]{1y.png}
    \caption{Выход замкнутой системы}
    \label{fig:1y}
\end{figure}


Время установления:
$$
T_s=k_sT,\ k_s=\min{\left\{ k:\, |y[m]-y_\infty|\leq\varepsilon\,\forall m\geq k \right\}}
$$


С момента $k=2$ значения равны нулю:
$$
\forall\varepsilon>0,m\geq k:\,|x[m]-0|\leq\varepsilon\Rightarrow T_s=2T=2
$$


Условие оптимальности по времени выполнилось:
$$
T_s=2\leq nT=2,
$$
где $n=2$ -- порядок системы.


\section{Проектирование дискретных следящих регуляторов}
Исходные данные:
\begin{align*}
    &g_0=2.12,\\
    &g_1=0,\\
    &A_g=0,\\
    &\omega_g=0,\\
    &g\left( k \right)=g_0+g_1kT=2.12
\end{align*}


Дискретная модель внешнего воздействия описывается в пространстве состояний системой разностных уравнений:
$$
\begin{cases}
    \xi(k+1)=\Gamma\xi(k),\\
    y(k)=H\xi(k),
\end{cases}
$$
где $\xi(k)$ -- $n$-мерный вектор состояния дискретного командного генератора,
$\Gamma_{n\times n}$ -- матрица динамических свойств дискретного генератора,
$H_{m=1\times n}$ -- матрица выхода модели.


Так как $g(k)=const.$, то $n=1$ -- матрицы $\Gamma,H$ -- константы.


Пусть:
$$
\Gamma=1,\ H=1,\ \xi(k)=g(k)
$$


Тогда генератор задающего воздействия:
$$
\begin{cases}
    \xi(k+1)=\xi(k),\\
    g(k)=\xi(k)
\end{cases}
$$


Начальное условие:
$$
\xi(0)=2.12
$$

\newpage
Схема моделирования генератора:
\begin{figure}[H]
    \centering
    \includegraphics[scale=0.35]{sch2.png}
    \caption{Схема моделирования генератора}
    \label{fig:sch2}
\end{figure}


Выполним моделирование генератора:
\begin{figure}[H]
    \centering
    \includegraphics[scale=0.8]{2g.png}
    \caption{Выход генератора}
    \label{fig:2g}
\end{figure}


Регулятор со встроенной моделью:
$$
\begin{cases}
    \eta(m+1)=\Gamma\eta(m)+B_\eta e(m),\\
    u(m)=k_1e(m)+K_\eta \eta(m)-k_2x_2(m)-...-k_nx_n(m),
\end{cases}
$$
где $\eta\in\mathbb{R}^q$ -- вектор состояния встроенной модели,
имеющий размерность вектора состояния внешних воздействий;
$\Gamma$ -- матрица размерности $q\times q$, совпадающая
с матрицей модели внешних воздействий;
$B_\eta$ -- матрица входов встроенной модели.


Расширенная модель ОУ (объединение уравнения движения объекта с уравнением встроенной модели):
$$
\begin{cases}
    \eta(m+1)=\Gamma\eta(m)+B_\eta g(m)-B_\eta y(m),\\
    x(m+1)=Ax(m)+Bu(m),\\
    y(m)=Cx(m)
\end{cases}
$$


Расширенный вектор состояния:
$$
\begin{bmatrix}
    \eta(m+1)\\
    x(m+1)
\end{bmatrix}=\begin{bmatrix}
    \Gamma&-B_\eta C\\0&A
\end{bmatrix}\begin{bmatrix}
    \eta(m)\\ x(m)
\end{bmatrix}+\begin{bmatrix}
    0\\ B
\end{bmatrix}u(m)+\begin{bmatrix}
    B_\eta\\0
\end{bmatrix}g(m)
$$


Матрицы:
$$
\bar{A}=\begin{bmatrix}
    \Gamma&-B_\eta C\\
    0&A
\end{bmatrix},\
\bar{B}=\begin{bmatrix}
    0\\ B
\end{bmatrix},\ \bar{B}_1=\begin{bmatrix}
    B_\eta \\ 0
\end{bmatrix},\ \bar{K}=\begin{bmatrix}
    -K_\eta& K
\end{bmatrix}
$$


Уравнения движения:
$$
\begin{cases}
    \bar{x}(m+1)=\bar{A}\bar{x}(m)+\bar{B}u(m)+\bar{B}_{1}g(m),\\
    u(m)=k_1g(m)-\bar{K}\bar{x}(m)
\end{cases}
$$


Замкнутая система:
$$
\bar{x}(m+1)=\left( \bar{A}-\bar{B}\bar{K} \right)\bar{x}(m)+\left( \bar{B}_{1}+k_1\bar{B} \right)g(m)=\bar{F}x(m)+\bar{B}_gg(m),
$$
где $\bar{F}$ -- матрица размерности
$\left( n+q \right)\times\left( n+q \right)$, определяющая
динамические свойства замкнутой системы;
$\bar{B}_g$
-- матрица входов по задающему
воздействию размерности $\left( n+q \right)\times1$.


Пара $\left( \Gamma,B_\eta \right)$ управляема при любом ненулевом $B_\eta$:
$$
B_\eta=1
$$


Условие управляемости $\left( \Gamma,B_\eta \right)$ выполнено.


Так как пара $\left( A,B \right)$ ОУ полностью управляема,
пара $\left( C,A \right)$ полностью наблюдаема
и пара $\left( \Gamma,B_\eta \right)$ полностью управляема,
то пара $\left( \bar{A},\bar{B} \right)$ полностью управляема.


В таком случае выбором матрицы $\bar{K}$
можно обеспечить произвольные желаемые корни
характеристического полинома или коэффициенты
уравнения замкнутой системы.


Оптимальная по быстродействию система при корнях
характеристического полинома $z_i^*=0$:
$$
\bar{\Gamma}=\begin{bmatrix}
    0&1&0\\
    0&0&1\\
    0&0&0
\end{bmatrix},\ \bar{H}=\begin{bmatrix}
    1&0&0
\end{bmatrix}
$$


Пара $\left( \bar{H},\bar{\Gamma} \right)$ полностью наблюдаема по признаку Жордана.


\section{Построение регуляторов для объектов с неполной информацией}
Исходные данные: четный вариант -- 
устройство оценки полной размерности.


\dots


\section{Вывод}
\dots
\end{document}