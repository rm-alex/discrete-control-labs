\documentclass[a4paper,14pt]{extarticle}

\usepackage[T2A]{fontenc}
\usepackage[utf8]{inputenc}
\usepackage[english, russian]{babel}

\usepackage[left=30mm, right=10mm, top=20mm, bottom=20mm]{geometry}

\usepackage{tempora}
\usepackage{setspace}
\onehalfspacing

\usepackage{titlesec}
\titleformat{\section}[block]{\bfseries\centering\MakeUppercase}{\thesection.}{1em}{}
\titleformat{\subsection}[block]{\bfseries}{\thesubsection.}{1em}{}
\titleformat{\subsubsection}[block]{\bfseries}{\thesubsubsection.}{1em}{}

\renewcommand{\contentsname}{\hfill \textbf{СОДЕРЖАНИЕ} \hfill\null}

\usepackage{indentfirst}
\setlength{\parindent}{1.25cm}

\usepackage{amsmath, amsfonts, amssymb}
\usepackage{graphicx}
\usepackage{caption}
\usepackage{subcaption}
\usepackage{float}
\usepackage{tikz}
\usetikzlibrary{patterns}
\usepackage{cmap}
\usepackage{hyperref}
\usepackage{xcolor}
\usepackage{listings}

\definecolor{LightGray}{gray}{0.7}

\lstdefinestyle{code}{
    language=Python, % change if needed
    basicstyle=\small\ttfamily,
    numbers=left,
    numberstyle=\small\color{LightGray},
    stepnumber=1,
    numbersep=5pt,
    backgroundcolor=\color{white},
    showspaces=false,
    showstringspaces=false,
    showtabs=false,
    tabsize=4,
    captionpos=b,
    breaklines=true,
    breakatwhitespace=false,
    frame=single,
    rulecolor=\color{LightGray},
    linewidth=\linewidth,
    keywordstyle=\color{blue}\bfseries,
    commentstyle=\color{green!40!black},
    stringstyle=\color{violet},
    escapeinside={\%*}{*)},
    xleftmargin=10pt,
    xrightmargin=10pt,
    framexleftmargin=0pt,
    framexrightmargin=0pt
}
\lstset{style=code}

\hypersetup{
    colorlinks=true,
    linkcolor=blue,
    filecolor=magenta,
    urlcolor=cyan,
    pdftitle={ncs1},
    pdfauthor={Rumyantsev Alexey},
    pdfsubject={control},
    pdfkeywords={LaTeX, PDF},
    pdfpagemode=FullScreen,
}

\graphicspath{{src/images/}}

\begin{document}

\begin{titlepage}
    \begin{center}
        МИНИСТЕРСТВО НАУКИ И ВЫСШЕГО ОБРАЗОВАНИЯ РОССИЙСКОЙ ФЕДЕРАЦИИ\\
        \vspace*{2.5mm}
        Федеральное государственное автономное образовательное учреждение высшего образования
        «НАЦИОНАЛЬНЫЙ ИССЛЕДОВАТЕЛЬСКИЙ УНИВЕРСИТЕТ ИТМО»\\
        \vspace*{2.5mm}
        \textbf{ФАКУЛЬТЕТ СИСТЕМ УПРАВЛЕНИЯ И РОБОТОТЕХНИКИ}
        \vfill

        {\large ОТЧЕТ ПО ЛАБОРАТОРНОЙ РАБОТЕ №4}\\
        {\large по дисциплине}\\
        {\large\bfseries «ДИСКРЕТНЫЕ СИСТЕМЫ УПРАВЛЕНИЯ»}\\
        {\large на тему}\\
        {\large\bfseries «АПЕРИОДИЧЕСКИЙ, ДАЛИНА И С ЗАДАННЫМ РАСПОЛОЖЕНИЕМ ПОЛЮСОВ РЕГУЛЯТОРЫ»}\\
        Вариант 20
        \vfill

        \begin{flushright}
            Выполнил: студент гр. R3441\\
            Румянцев А. А.\medskip\\

            Проверил: преподаватель\\
            Краснов А. Ю.
        \end{flushright}
        \vfill

        Санкт-Петербург\\
        2025
    \end{center}
\end{titlepage}

\setcounter{page}{2}
\tableofcontents
\newpage

\section{Исходные данные}
Исходные данные варианта 20:
\begin{table}[h!]
\centering
\begin{tabular}{|c|c|c|c|c|c|}
\hline
$T$ & $a$ & $b$ &$\zeta$ & $\omega_d$ & $K_v$\\ \hline
0.55 & 1.1 &10 & 0.35 & 4 & 0.1 \\ \hline
\end{tabular}
\end{table}


\section{Выполнение работы}
ОУ задан непрерывной передаточной функцией
$$
G(s)=\frac{e^{-as}}{1+bs}=\frac{e^{-1.1s}}{1+10s}
$$


\subsection{Апериодический регулятор}
Синтезируем для непрерывного ОУ апериодический
регулятор при периоде дискретизации $T=1$.


Желаемая передаточная функция системы:
$$
T(z)=\frac{Y(z)}{R(z)}=z^{-k},\ k\geq1
$$


Замкнутая система:
$$
T(z)=\frac{D(z)G(z)H}{1+D(z)G(z)H}
$$


Передаточная функция регулятора:
$$
D(z)=\frac{1}{HG(z)}\frac{T(z)}{1-T(z)}=\frac{1}{HG(z)}\frac{z^{-k}}{1-z^{-k}}
$$


Экспонента $e^{-as}$ описывает задержку
в непрерывной системе.


При дискретизации представляем задержку через
$z^{-n_d}$, где:
$$
n_d=\frac{a}{T}
$$


Если $n_d$ целое, то задержка точная в дискретной модели.


Если $n_d$ дробное, то округляем задержку до ближайшего
целого.


Таким образом,
$$
n_d=\frac{1.1}{1}=1.1\Rightarrow n_d\sim 1\Rightarrow e^{-1.1s}\sim e^{-s}\Rightarrow Z\left\{ e^{-1.1s} \right\}\sim Z\left\{ e^{-s} \right\}=z^{-1}
$$


Дискретная передаточная функция ОУ с ЭНП:
\begin{align*}
    HG(z)&=Z\left\{ \frac{1-e^{-sT}}{s}G(s) \right\}=\left( 1-z^{-1} \right)Z\left\{ \frac{e^{-1.1s}}{s\left( 1+10s \right)} \right\}=\\
    &=\left( 1-z^{-1} \right)z^{-1}Z\left\{ \frac{1}{s\left( 1+10s \right)} \right\}=\left( 1-z^{-1} \right)z^{-1}Z\left\{ \frac{0.1}{s\left( s+0.1 \right)} \right\}=\\
    &=\left( 1-z^{-1} \right)z^{-1}\frac{z\left( 1-e^{-0.1} \right)}{\left( z-1 \right)\left( z-e^{-0.1} \right)}=z^{-2}\frac{1-e^{-0.1}}{1-e^{-0.1}z^{-1}}=\frac{0.095z^{-2}}{1-0.905z^{-1}}=\\
    &=\frac{0.095}{z^2-0.905z}
\end{align*}


Подставим в передаточную функцию регулятора:
$$
D(z)=\frac{1-0.905z^{-1}}{0.095z^{-2}}\frac{z^{-k}}{1-z^{-k}}
$$


Из условия физической реализуемости $k\geq2$ выберем $k=2$, тогда:
$$
D(z)=\frac{1-0.905z^{-1}}{0.095z^{-2}}\frac{z^{-2}}{1-z^{-2}}=
\frac{z^2-0.905z}{0.095\left( z^2-1 \right)}
$$


Модель системы в симулинк:
\begin{figure}[H]
    \centering
    \includegraphics[scale=0.25]{sch1.png}
    \caption{Схема моделирования системы с апериодическим регулятором}
    \label{fig:sch1}
\end{figure}


\newpage
Графики управления и выхода системы при ступенчатом задающем воздействии $r(m)$ при $T=1,k=\left[ 2,3 \right]$:
\begin{figure}[H]
    \centering
    \includegraphics[scale=1]{r.png}
    \caption{Ступенчатое задающее воздействие $r(m)$}
    \label{fig:r}
\end{figure}
\begin{figure}[H]
    \centering
    \includegraphics[scale=1]{1y.png}
    \caption{Выход системы с апериодическим регулятором, $T=1,k=2$}
    \label{fig:1y}
\end{figure}
\begin{figure}[H]
    \centering
    \includegraphics[scale=1]{1u.png}
    \caption{Апериодический регулятор, $T=1,k=2$}
    \label{fig:1u}
\end{figure}


\begin{figure}[H]
    \centering
    \includegraphics[scale=1]{1y_k3.png}
    \caption{Выход системы с апериодическим регулятором, $T=1,k=3$}
    \label{fig:1y_k3}
\end{figure}
\begin{figure}[H]
    \centering
    \includegraphics[scale=1]{1u_k3.png}
    \caption{Апериодический регулятор, $T=1,k=3$}
    \label{fig:1u_k3}
\end{figure}


Выход системы отстает на $m=2$ шага от входа при $k=2$,
при $k=3$ на $m=3$ шага, т.е. отставание на $k$.


\subsection{Регулятор Далина}
Синтезируем для непрерывного ОУ регулятор
Далина при периоде дискретизации $T=1$.


Регулятор Далина -- модификация апериодического
регулятора, имеющая более плавный экспоненциальный
отклик.


Желаемое поведение системы в $s$-плоскости:
$$
Y(s)=\frac{1}{s}\frac{e^{-as}}{1+bs}=\frac{1}{s}\frac{e^{-1.1s}}{1+10s}
$$


Параметры $a,b$ определяют численные параметры
желаемого поведения выходной величины.


$Z$-преобразование желаемой реакции
системы при $a=kT$:
$$
Y(z)=\frac{z^{-k-1}\left( 1-e^{-T/b} \right)}{\left( 1-z^{-1} \right)\left( 1-e^{-T/b}z^{-1} \right)}
$$


Желаемая передаточная функция замкнутой системы:
$$
T(z)=\frac{Y(z)}{R(z)}=\frac{z^{-k-1}\left( 1-e^{-T/b} \right)\left( 1-z^{-1} \right)}{\left( 1-z^{-1} \right)\left( 1-e^{-T/b}z^{-1} \right)}=\frac{z^{-k-1}\left( 1-e^{-T/b} \right)}{1-e^{-T/b}z^{-1}}
$$


Передаточная функция регулятора:
$$
D(z)=\frac{1}{HG(z)}\frac{z^{-k-1}\left( 1-e^{-T/b} \right)}{1-e^{-T/b}z^{-1}-\left( 1-e^{-T/b} \right)z^{-k-1}}
$$


Передаточная функция ОУ с ЭНП выражается аналогично
пункту с апериодическим регулятором:
$$
e^{-1.1s}\sim e^{-s},\ HG(z)=Z\left\{ \frac{1-e^{-sT}}{s}G(s) \right\}=\frac{0.095z^{-2}}{1-0.905z^{-1}}=\frac{0.095}{z^2-0.905z}
$$


Передаточная функция регулятора:
\begin{align*}
    D(z)&=\frac{1-0.905z^{-1}}{0.095z^{-2}}\frac{z^{-k-1}\left( 1-e^{-0.1} \right)}{1-e^{-0.1}z^{-1}-\left( 1-e^{-0.1} \right)z^{-k-1}}=\\
    &=\frac{1-0.905z^{-1}}{0.095z^{-2}}\frac{0.095z^{-k-1}}{1-0.905z^{-1}-0.095z^{-k-1}}
\end{align*}


С учетом требования физической реализуемости $k\geq1$ положим $k=1$:
\begin{align*}
    D(z)&=\frac{1-0.905z^{-1}}{0.095z^{-2}}\frac{0.095z^{-2}}{1-0.905z^{-1}-0.095z^{-2}}=\\
    &=\frac{1-0.905z^{-1}}{1-0.905z^{-1}-0.095z^{-2}}=\frac{0.095z^2-0.086z}{0.095z^2-0.086z-0.009}
\end{align*}


Схема моделирования замкнутой системы:
\begin{figure}[H]
    \centering
    \includegraphics[scale=0.25]{sch2.png}
    \caption{Схема моделирования системы с регулятором Далина}
    \label{fig:sch2}
\end{figure}


Графики управления и выхода системы при ступенчатом задающем воздействии $r(m)$ (см. рис. (\ref{fig:r})) при $T=1,k=\left[ 1,3 \right]$:
\begin{figure}[H]
    \centering
    \includegraphics[scale=1]{2y.png}
    \caption{Выход системы с регулятором Далина, $T=1,k=1$}
    \label{fig:2y}
\end{figure}
\begin{figure}[H]
    \centering
    \includegraphics[scale=1]{2u.png}
    \caption{Регулятор Далина, $T=1,k=1$}
    \label{fig:2u}
\end{figure}


\begin{figure}[H]
    \centering
    \includegraphics[scale=1]{2y_k3.png}
    \caption{Выход системы с регулятором Далина, $T=1,k=3$}
    \label{fig:2y_k3}
\end{figure}
\begin{figure}[H]
    \centering
    \includegraphics[scale=1]{2u_k3.png}
    \caption{Регулятор Далина, $T=1,k=3$}
    \label{fig:2u_k3}
\end{figure}


Величина управления меньше по сравнению с апериодическим
регулятором, но выход сходится дольше к устоявшемуся значению.
Отставание на $k+1$.


\subsection{Регулятор с заданным расположением полюсов}
Дискретизованный ОУ с ЭНП задан передаточной функцией:
$$
HG(z)=\frac{0.03\left( z+0.75 \right)}{z^2-1.5z+0.5}
$$


Разработаем дискретный регулятор,
обеспечивающий заданное расположение
полюсов замкнутой системы так, чтобы
ее отклик был колебательным с
декрементом затухания $\zeta=0.35$ и частотой колебаний $\omega_d=4$. Установившаяся
ошибка для ступенчатого входа должна быть
равна нулю. Установившаяся ошибка для
линейно нарастающего входа должна быть
равна $K_v=0.1$. Период дискретизации $T=0.55$.


Полюса передаточной функции:
$$
z_{1,2}=e^{-\zeta\omega_n T\pm j\omega_nT\sqrt{1-\zeta^2}}=
e^{-\zeta \omega_n T}\left( \cos{\left( \omega_n T\sqrt{1-\zeta^2} \right)}\pm j\sin{\left( \omega_n T\sqrt{1-\zeta^2} \right)} \right)
$$


Связь демпфированного синусоидального частотного компонента с собственной частотой:
$$
\omega_d=\omega_n\sqrt{1-\zeta^2}\Rightarrow \omega_n=\frac{\omega_d}{\sqrt{1-\zeta^2}}=\frac{4}{\sqrt{1-0.35^2}}\approx4.27
$$


Тогда:
\begin{align*}
    z_{1,2}&=e^{-\zeta \omega_n T}\left( \cos{\left( \omega_d T \right)}\pm j\sin{\left( \omega_d T \right)} \right)=\\
    &=e^{-0.35\cdot4.27\cdot0.55}\left( \cos{\left( 4\cdot0.55 \right)}\pm j\sin{\left(4\cdot0.55\right)} \right)=\\
    &=-0.259\pm 0.355j
\end{align*}


Желаемая передаточная функция замкнутой системы:
\begin{align*} 
    T(z)&=\frac{b_0+b_1z^{-1}+b_2z^{-2}+...+b_nz^{-n}}{\left( z-\left( -0.259+0.355j \right)\right)\left( z-\left( -0.259-0.355j \right) \right)}=\\
    &=\frac{b_0+b_1z^{-1}+b_2z^{-2}+...+b_nz^{-n}}{1+0.518z^{-1}+0.193z^{-2}}
\end{align*}


Физическая реализуемость: $b_0=0,b_1,b_2\neq0$.
Передаточная функция:
$$
T(z)=\frac{b_1z^{-1}+b_2z^{-2}}{1+0.518z^{-1}+0.193z^{-2}}
$$


Найдем коэффициенты числителя.


Установившаяся ошибка:
$$
E(z)=R(z)\left( 1-T(z) \right)
$$


Для единичного ступенчатого входного
сигнала установившуюся ошибку можно
определить с помощью теоремы о
конечном значении:
$$
E_{ss}=\lim\limits_{z\to1}\frac{z-1}{z}\frac{z}{z-1}[v]=1-T(1)
$$


Установившаяся ошибка примет нулевое значение при $T(1)=1$:
$$
T(z=1)=\frac{b_1+b_2}{1.711}=1\Rightarrow b_1+b_2=1.711,
$$
$$
T(z)=\frac{b_1z+b_2}{z^2+0.518z+0.193}
$$


Пусть $K_v$ -- добротность
по скорости замкнутой системы,
тогда установившаяся ошибка
при линейно нарастающем входном
воздействии:
$$
E_{ss}=\lim\limits_{z\to1}\frac{z-1}{z}\frac{Tz}{\left( z-1 \right)^2}\left[ 1-T(z) \right]=\frac{1}{K_v}
$$
или, используя правило Лопиталя
$$
\left.\frac{dT}{dz}\right|_{z=1}=-\frac{1}{K_v T}
$$


Получим:
$$
\left.\frac{dT}{dz}\right|_{z=1}=\left.\frac{b_1\left( z^2+0.518z+0.193 \right)-\left( b_1z+b_2 \right)\left( 2z+0.518 \right)}{\left( z^2+0.518z+0.193 \right)^2}\right|_{z=1}=-\frac{1}{K_v T}
$$


Подставим значения:
$$
\frac{-0.807b_1-2.518b_2}{2.928}=-18.182
$$


Составим систему и найдем коэффициенты числителя:
$$
\begin{cases}
    -0.807b_1-2.518b_2=-53.236,\\
    b_1+b_2=1.711
\end{cases}\Rightarrow\begin{cases}
    b_1=1.711-b_2,\\
    -1.711b_2=-51.855,
\end{cases}
$$
$$
\begin{cases}
    b_1=-28.596,\\
    b_2=30.307
\end{cases}
$$


Передаточная функция:
$$
T(z)=\frac{-28.596z^{-1}+30.307z^{-2}}{1+0.518z^{-1}+0.193z^{-2}}=\frac{-28.596z+30.307}{z^{2}+0.518z+0.193}
$$


Передаточная функция регулятора:
$$
D(z)=\frac{1}{HG(z)}\frac{T(z)}{1-T(z)},
$$
$$
D(z)=\frac{z^2-1.5z+0.5}{0.03\left( z+0.75 \right)}\cdot\frac{-28.596z+30.307}{z^2+29.114z-30.114},\\
$$
$$
D(z)=\frac{-28.596z^3+73.201z^2-59.759z+15.154}{0.03z^3+0.896z^2-0.248z-0.678}
$$


Схема моделирования системы с регулятором с заданным расположением полюсов:
\begin{figure}[H]
    \centering
    \includegraphics[scale=0.3]{sch3.png}
    \caption{Схема моделирования системы с fixed-pole регулятором}
    \label{fig:sch3}
\end{figure}


\newpage
Графики управления, выхода системы и ошибки при ступенчатом (см. рис. (\ref{fig:r}))
и линейно нарастающем задающих воздействиях $r(m),\tilde{r}(m)$ при $T=0.55$:
\begin{figure}[H]
    \centering
    \includegraphics[scale=1]{3y_r.png}
    \caption{Выход системы с fixed-pole регулятором, $T=0.55$, вход $r(m)$}
    \label{fig:3y_r}
\end{figure}
\begin{figure}[H]
    \centering
    \includegraphics[scale=1]{3e_r.png}
    \caption{Ошибка с fixed-pole регулятором, $T=0.55$, вход $r(m)$}
    \label{fig:3e_r}
\end{figure}
\begin{figure}[H]
    \centering
    \includegraphics[scale=1]{3u_r.png}
    \caption{Fixed-pole регулятор, $T=0.55$, вход $r(m)$}
    \label{fig:3u_r}
\end{figure}


\begin{figure}[H]
    \centering
    \includegraphics[scale=1]{r2.png}
    \caption{Линейно нарастающее задающее воздействие $\tilde{r}(m)$}
    \label{fig:r2}
\end{figure}
\begin{figure}[H]
    \centering
    \includegraphics[scale=1]{3y_r2.png}
    \caption{Выход системы с fixed-pole регулятором, $T=0.55$, вход $\tilde{r}(m)$}
    \label{fig:3y_r2}
\end{figure}
\begin{figure}[H]
    \centering
    \includegraphics[scale=1]{3e_r2.png}
    \caption{Ошибка с fixed-pole регулятором, $T=0.55$, вход $\tilde{r}(m)$}
    \label{fig:3e_r2}
\end{figure}
\begin{figure}[H]
    \centering
    \includegraphics[scale=1]{3u_r2.png}
    \caption{Fixed-pole регулятор, $T=0.55$, вход $\tilde{r}(m)$}
    \label{fig:3u_r2}
\end{figure}


Ошибка при ступенчатом входном воздействии стремится к нулю.


Ошибка при линейно нарастающем входном воздействии с $A=1$ стремится к 10.
Проверим теоретически:
$$
E_{ss}=\frac{A}{K_v}=\frac{1}{0.1}=10
$$


\section{Вывод}
В ходе выполнения лабораторной работы
были исследованы апериодический, Далина и
с заданным расположением полюсов регуляторы.
В каждом случае было проведено моделирование системы.
Результаты подтверждают корректность выполненных расчетов.
\end{document}