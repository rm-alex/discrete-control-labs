\documentclass[a4paper,14pt]{extarticle}

\usepackage[T2A]{fontenc}
\usepackage[utf8]{inputenc}
\usepackage[english, russian]{babel}

\usepackage[left=30mm, right=10mm, top=20mm, bottom=20mm]{geometry}

\usepackage{tempora}
\usepackage{setspace}
\onehalfspacing

\usepackage{titlesec}
\titleformat{\section}[block]{\bfseries\centering\MakeUppercase}{\thesection.}{1em}{}
\titleformat{\subsection}[block]{\bfseries}{\thesubsection.}{1em}{}
\titleformat{\subsubsection}[block]{\bfseries}{\thesubsubsection.}{1em}{}

\renewcommand{\contentsname}{\hfill \textbf{СОДЕРЖАНИЕ} \hfill\null}

\usepackage{indentfirst}
\setlength{\parindent}{1.25cm}

\usepackage{amsmath, amsfonts, amssymb}
\usepackage{graphicx}
\usepackage{caption}
\usepackage{subcaption}
\usepackage{float}
\usepackage{tikz}
\usetikzlibrary{patterns}
\usepackage{cmap}
\usepackage{hyperref}
\usepackage{xcolor}
\usepackage{listings}

\definecolor{LightGray}{gray}{0.7}

\lstdefinestyle{code}{
    language=Python, % change if needed
    basicstyle=\small\ttfamily,
    numbers=left,
    numberstyle=\small\color{LightGray},
    stepnumber=1,
    numbersep=5pt,
    backgroundcolor=\color{white},
    showspaces=false,
    showstringspaces=false,
    showtabs=false,
    tabsize=4,
    captionpos=b,
    breaklines=true,
    breakatwhitespace=false,
    frame=single,
    rulecolor=\color{LightGray},
    linewidth=\linewidth,
    keywordstyle=\color{blue}\bfseries,
    commentstyle=\color{green!40!black},
    stringstyle=\color{violet},
    escapeinside={\%*}{*)},
    xleftmargin=10pt,
    xrightmargin=10pt,
    framexleftmargin=0pt,
    framexrightmargin=0pt
}
\lstset{style=code}

\hypersetup{
    colorlinks=true,
    linkcolor=blue,
    filecolor=magenta,
    urlcolor=cyan,
    pdftitle={ncs1},
    pdfauthor={Rumyantsev Alexey},
    pdfsubject={control},
    pdfkeywords={LaTeX, PDF},
    pdfpagemode=FullScreen,
}

\graphicspath{{src/images/}}

\begin{document}

\begin{titlepage}
    \begin{center}
        МИНИСТЕРСТВО НАУКИ И ВЫСШЕГО ОБРАЗОВАНИЯ РОССИЙСКОЙ ФЕДЕРАЦИИ\\
        \vspace*{2.5mm}
        Федеральное государственное автономное образовательное учреждение высшего образования
        «НАЦИОНАЛЬНЫЙ ИССЛЕДОВАТЕЛЬСКИЙ УНИВЕРСИТЕТ ИТМО»\\
        \vspace*{2.5mm}
        \textbf{ФАКУЛЬТЕТ СИСТЕМ УПРАВЛЕНИЯ И РОБОТОТЕХНИКИ}
        \vfill

        {\large ОТЧЕТ ПО ЛАБОРАТОРНОЙ РАБОТЕ №4}\\
        {\large по дисциплине}\\
        {\large\bfseries «ДИСКРЕТНЫЕ СИСТЕМЫ УПРАВЛЕНИЯ»}\\
        {\large на тему}\\
        {\large\bfseries «АПЕРИОДИЧЕСКИЙ, ДАЛИНА И С ЗАДАННЫМ РАСПОЛОЖЕНИЕМ ПОЛЮСОВ РЕГУЛЯТОРЫ»}\\
        Вариант 20
        \vfill

        \begin{flushright}
            Выполнил: студент гр. R3441\\
            Румянцев А. А.\medskip\\

            Проверил: преподаватель\\
            Краснов А. Ю.
        \end{flushright}
        \vfill

        Санкт-Петербург\\
        2025
    \end{center}
\end{titlepage}

\setcounter{page}{2}
\tableofcontents
\newpage

\section{Исходные данные}
Исходные данные варианта 20:
\begin{table}[h!]
\centering
\begin{tabular}{|c|c|c|c|c|c|}
\hline
$T$ & $a$ & $b$ &$\zeta$ & $\omega_d$ & $K_v$\\ \hline
0.55 & 1.1 &10 & 0.35 & 4 & 0.1 \\ \hline
\end{tabular}
\end{table}


\section{Выполнение работы}
ОУ задан непрерывной передаточной функцией
$$
G(s)=\frac{e^{-as}}{1+bs}=\frac{e^{-1.1s}}{1+10s}
$$


\subsection{Апериодический регулятор}
Синтезируем для непрерывного ОУ апериодический
регулятор при периоде дискретизации $T=1$.


Желаемая передаточная функция системы:
$$
T(z)=\frac{Y(z)}{R(z)}=z^{-k},\ k\geq1
$$


Замкнутая система:
$$
T(z)=\frac{D(z)G(z)H}{1+D(z)G(z)H}
$$


Передаточная функция регулятора:
$$
D(z)=\frac{1}{HG(z)}\frac{T(z)}{1-T(z)}=\frac{1}{HG(z)}\frac{z^{-k}}{1-z^{-k}}
$$


Экспонента $e^{-as}$ описывает задержку
в непрерывной системе.


При дискретизации представляем задержку через
$z^{-n_d}$, где:
$$
n_d=\frac{a}{T}
$$


Если $n_d$ целое, то задержка точная в дискретной модели.


Если $n_d$ дробное, то округляем задержку до ближайшего
целого.


Таким образом,
$$
n_d=\frac{1.1}{1}=1.1\Rightarrow n_d\sim 1\Rightarrow e^{-1.1s}\sim e^{-s}\Rightarrow Z\left\{ e^{-1.1s} \right\}\sim Z\left\{ e^{-s} \right\}=z^{-1}
$$


Дискретная передаточная функция ОУ с ЭНП:
\begin{align*}
    HG(z)&=Z\left\{ \frac{1-e^{-sT}}{s}G(s) \right\}=\left( 1-z^{-1} \right)Z\left\{ \frac{e^{-1.1s}}{s\left( 1+10s \right)} \right\}=\\
    &=\left( 1-z^{-1} \right)z^{-1}Z\left\{ \frac{1}{s\left( 1+10s \right)} \right\}=\left( 1-z^{-1} \right)z^{-1}Z\left\{ \frac{0.1}{s\left( s+0.1 \right)} \right\}=\\
    &=\left( 1-z^{-1} \right)z^{-1}\frac{z\left( 1-e^{-0.1} \right)}{\left( z-1 \right)\left( z-e^{-0.1} \right)}=z^{-2}\frac{1-e^{-0.1}}{1-e^{-0.1}z^{-1}}=\frac{0.095z^{-2}}{1-0.905z^{-1}}=\\
    &=\frac{0.095}{z^2-0.905z}
\end{align*}


Подставим в передаточную функцию регулятора:
$$
D(z)=\frac{1-0.905z^{-1}}{0.095z^{-2}}\frac{z^{-k}}{1-z^{-k}}
$$


Из условия физической реализуемости выберем $k=2$, тогда:
$$
D(z)=\frac{1-0.905z^{-1}}{0.095z^{-2}}\frac{z^{-2}}{1-z^{-2}}=
\frac{z^2-0.905z}{0.095\left( z^2-1 \right)}
$$


Модель системы в симулинк:
\begin{figure}[H]
    \centering
    \includegraphics[scale=0.25]{sch1.png}
    \caption{Схема моделирования системы с апериодическим регулятором}
    \label{fig:sch1}
\end{figure}


\newpage
Графики управления и выхода системы при ступенчатом задающем воздействии с начальным
значением 0, конечным 1, время шага 1; $T=1$:
\begin{figure}[H]
    \centering
    \includegraphics[scale=1]{1y.png}
    \caption{Выход системы с апериодическим регулятором}
    \label{fig:1y}
\end{figure}
\begin{figure}[H]
    \centering
    \includegraphics[scale=1]{1u.png}
    \caption{Апериодический регулятор}
    \label{fig:1u}
\end{figure}


\subsection{Регулятор Далина}
Синтезируем для непрерывного ОУ регулятор
Далина при периоде дискретизации $T=1$.


Регулятор Далина -- модификация апериодического
регулятора, имеющая более плавный экспоненциальный
отклик.


Желаемое поведение системы в $s$-плоскости:
$$
Y(s)=\frac{1}{s}\frac{e^{-as}}{1+bs}=\frac{1}{s}\frac{e^{-1.1s}}{1+10s}
$$


Параметры $a,b$ определяют численные параметры
желаемого поведения выходной величины.


$Z$-преобразование желаемой реакции
системы при $a=kT$:
$$
Y(z)=\frac{z^{-k-1}\left( 1-e^{-T/b} \right)}{\left( 1-z^{-1} \right)\left( 1-e^{-T/b}z^{-1} \right)}
$$


Желаемая передаточная функция замкнутой системы:
$$
T(z)=\frac{Y(z)}{R(z)}=\frac{z^{-k-1}\left( 1-e^{-T/b} \right)\left( 1-z^{-1} \right)}{\left( 1-z^{-1} \right)\left( 1-e^{-T/b}z^{-1} \right)}=\frac{z^{-k-1}\left( 1-e^{-T/b} \right)}{1-e^{-T/b}z^{-1}}
$$


Передаточная функция регулятора:
$$
D(z)=\frac{1}{HG(z)}\frac{z^{-k-1}\left( 1-e^{-T/b} \right)}{1-e^{-T/b}z^{-1}-\left( 1-e^{-T/b} \right)z^{-k-1}}
$$


Передаточная функция ОУ с ЭНП выражается аналогично
пункту с апериодическим регулятором:
$$
e^{-1.1s}\sim e^{-s},\ HG(z)=Z\left\{ \frac{1-e^{-sT}}{s}G(s) \right\}=\frac{0.095z^{-2}}{1-0.905z^{-1}}=\frac{0.095}{z^2-0.905z}
$$


Передаточная функция регулятора:
\begin{align*}
    D(z)&=\frac{1-0.905z^{-1}}{0.095z^{-2}}\frac{z^{-k-1}\left( 1-e^{-0.1} \right)}{1-e^{-0.1}z^{-1}-\left( 1-e^{-0.1} \right)z^{-k-1}}=\\
    &=\frac{1-0.905z^{-1}}{0.095z^{-2}}\frac{0.095z^{-k-1}}{1-0.905z^{-1}-0.095z^{-k-1}}
\end{align*}


С учетом требования физической реализуемости положим $k=1$:
\begin{align*}
    D(z)&=\frac{1-0.905z^{-1}}{0.095z^{-2}}\frac{0.095z^{-2}}{1-0.905z^{-1}-0.095z^{-2}}=\\
    &=\frac{1-0.905z^{-1}}{1-0.905z^{-1}-0.095z^{-2}}=\frac{0.095z^2-0.086z}{0.095z^2-0.086z-0.009}
\end{align*}


Схема моделирования замкнутой системы с дискретизированным ОУ
и регулятором:
\begin{figure}[H]
    \centering
    \includegraphics[scale=0.25]{sch2.png}
    \caption{Схема моделирования системы с регулятором Далина}
    \label{fig:sch2}
\end{figure}


Графики управления и выхода системы при ступенчатом задающем воздействии с начальным
значением 0, конечным 1, время шага 1; $T=1$:
\begin{figure}[H]
    \centering
    \includegraphics[scale=1]{2y.png}
    \caption{Выход системы с регулятором Далина}
    \label{fig:2y}
\end{figure}
\begin{figure}[H]
    \centering
    \includegraphics[scale=1]{2u.png}
    \caption{Регулятор Далина}
    \label{fig:2u}
\end{figure}


\section{Вывод}
...
\end{document}