\documentclass[a4paper,14pt]{extarticle}

\usepackage[T2A]{fontenc}
\usepackage[utf8]{inputenc}
\usepackage[english, russian]{babel}

\usepackage[left=30mm, right=10mm, top=20mm, bottom=20mm]{geometry}

\usepackage{tempora}
\usepackage{setspace}
\onehalfspacing

\usepackage{titlesec}
\titleformat{\section}[block]{\bfseries\centering\MakeUppercase}{\thesection.}{1em}{}
\titleformat{\subsection}[block]{\bfseries}{\thesubsection.}{1em}{}
\titleformat{\subsubsection}[block]{\bfseries}{\thesubsubsection.}{1em}{}

\renewcommand{\contentsname}{\hfill \textbf{СОДЕРЖАНИЕ} \hfill\null}

\usepackage{indentfirst}
\setlength{\parindent}{1.25cm}

\usepackage{amsmath, amsfonts, amssymb}
\usepackage{graphicx}
\usepackage{caption}
\usepackage{subcaption}
\usepackage{float}
\usepackage{tikz}
\usetikzlibrary{patterns}
\usepackage{cmap}
\usepackage{hyperref}
\usepackage{xcolor}
\usepackage{listings}

\definecolor{LightGray}{gray}{0.7}

\lstdefinestyle{code}{
    language=Python, % change if needed
    basicstyle=\small\ttfamily,
    numbers=left,
    numberstyle=\small\color{LightGray},
    stepnumber=1,
    numbersep=5pt,
    backgroundcolor=\color{white},
    showspaces=false,
    showstringspaces=false,
    showtabs=false,
    tabsize=4,
    captionpos=b,
    breaklines=true,
    breakatwhitespace=false,
    frame=single,
    rulecolor=\color{LightGray},
    linewidth=\linewidth,
    keywordstyle=\color{blue}\bfseries,
    commentstyle=\color{green!40!black},
    stringstyle=\color{violet},
    escapeinside={\%*}{*)},
    xleftmargin=10pt,
    xrightmargin=10pt,
    framexleftmargin=0pt,
    framexrightmargin=0pt
}
\lstset{style=code}

\hypersetup{
    colorlinks=true,
    linkcolor=blue,
    filecolor=magenta,
    urlcolor=cyan,
    pdftitle={ncs1},
    pdfauthor={Rumyantsev Alexey},
    pdfsubject={control},
    pdfkeywords={LaTeX, PDF},
    pdfpagemode=FullScreen,
}

\graphicspath{{src/images/}}

\begin{document}

\begin{titlepage}
    \begin{center}
        МИНИСТЕРСТВО НАУКИ И ВЫСШЕГО ОБРАЗОВАНИЯ РОССИЙСКОЙ ФЕДЕРАЦИИ\\
        \vspace*{2.5mm}
        Федеральное государственное автономное образовательное учреждение высшего образования
        «НАЦИОНАЛЬНЫЙ ИССЛЕДОВАТЕЛЬСКИЙ УНИВЕРСИТЕТ ИТМО»\\
        \vspace*{2.5mm}
        \textbf{ФАКУЛЬТЕТ СИСТЕМ УПРАВЛЕНИЯ И РОБОТОТЕХНИКИ}
        \vfill

        {\large ОТЧЕТ ПО ЛАБОРАТОРНОЙ РАБОТЕ №1}\\
        {\large по дисциплине}\\
        {\large\bfseries «ДИСКРЕТНЫЕ СИСТЕМЫ УПРАВЛЕНИЯ»}\\
        {\large на тему}\\
        {\large\bfseries «УСТОЙЧИВОСТЬ ДИСКРЕТНЫХ СИСТЕМ. ПОСТРОЕНИЕ ДИСКРЕТНЫХ КОМАНДНЫХ ГЕНЕРАТОРОВ»}\\
        Вариант 20
        \vfill

        \begin{flushright}
            Выполнил: студент гр. R3441\\
            Румянцев А. А.\medskip\\

            Проверил: преподаватель\\
            Краснов А. Ю.
        \end{flushright}
        \vfill

        Санкт-Петербург\\
        2025
    \end{center}
\end{titlepage}

\setcounter{page}{2}
\tableofcontents
\newpage


\section{Исследование влияния дискретного элемента на непрерывную систему}
Исходные данные для задания:
\begin{align*}
    &T=0.25\text{ с},\\
    &K_{CO}=8.8
\end{align*}


$T$ -- период дискретизации, $K_{CO}$
-- коэффициент передачи ОУ.


Схема симуляции:
\begin{figure}[H]
    \centering
    \includegraphics[scale=0.25]{sch1.png}
    \caption{Схема симуляции}
    \label{fig:sch1}
\end{figure}


Непрерывная система:
$$
y(t)=\int\limits_0^tu\left( \tau \right)d\,\tau
$$


Возьмем  $\tau\in\left[ t_k=kT,t_{k+1}=\left( k+1 \right)T \right)$ -- некоторый шаг дискретизации.
Приращение интеграла за этот интервал:
$$
y\left( \left( k+1 \right)T \right)-y\left( kT \right)=\int\limits_{kT}^{\left( k+1 \right)T}u\left( \tau \right)\,d\tau
$$


Обозначения:
$$
y_k=y\left( kT \right),\ u_k=u\left( kT \right)
$$


Дискретная система:
$$
y_{k+1}-y_k=Tu_k,
$$
$$
Tu_k=\int\limits_{kT}^{\left( k+1 \right)T}u_k\,d\tau\approx\int\limits_{kT}^{\left( k+1 \right)T}u\left( \tau \right)\,d\tau,
$$
$$
u_k=K_{CO}K_{FB}\,e_k=K\left( r_k-y_k \right),
$$
где $r_k$ -- вход системы.


Приведем систему к виду:
$$
y_{k+1}=ay_k+br_k,
$$
$$
y_{k+1}=y_k+TK\left( r_k-y_k \right)=\left( 1-TK \right)y_k+TKr_k
$$


Пусть вход -- единичная ступень $r_k=1\,\forall k\geq0$, тогда разностное уравнение:
$$
y_{k+1}=ay_k+b
$$


Решение рекуррентного уравнения -- сумма общего $(h)$
и частного $(p)$ решений:
$$
y^{\left( h \right)}_{k+1}=ay^{\left( h \right)}_k\Rightarrow y^{\left( h \right)}_k=Ca^k,
$$
$$
y^{\left( p \right)}=ay^{\left( p \right)}+b\Rightarrow y^{\left( p \right)}\left( 1-a \right)=b\Rightarrow y^{\left( p \right)}=\frac{b}{1-a}=const.,
$$
$$
y_k=Ca^k+\frac{b}{1-a}
$$


При начальном условии $y_0=0$:
$$
Ca^0+\frac{b}{1-a}=0\Rightarrow C=-\frac{b}{1-a}
$$


Тогда:
$$
y_k=b\,\frac{1-a^k}{1-a}
$$


Переходный процесс описывается $a_k$.


При $k\to\infty,|a|<1:a^k\to0$, остается
установившееся значение:
$$
y_{\infty}=\frac{b}{1-a}
$$


Чем меньше $|a|$, тем быстрее убывает $a^k$ --
быстрее достигается установившееся значение.


Тогда, максимальная скорость сходимости будет при $a=0$.


Система устойчива при $|a|<1$:
$$
|1-TK|<1\Rightarrow 0<K_{FB}<\frac{2}{TK_{CO}}\approx0.909091
$$


Рассмотрим границы устойчивости.


Нейтральная устойчивость:
$$
a=1\Rightarrow 1-TK=1\Rightarrow K_{FB}=0
$$


При нейтральной устойчивости выход
повторяет начальное значение $y_0$,
задаваемое в интеграторе.


Колебательная устойчивость:
$$
a=-1\Rightarrow 1-TK=-1\Rightarrow K_{FB}=\frac{2}{TK_{CO}}\approx 0.909091
$$


При колебательной устойчивости
система будет максимально колебательной
без затухания.


Рассмотрим подробнее поведение системы при различных $a$:
\begin{align*}
    &\text{При }a=-1\text{ -- колебательная устойчивость},\\
    &\text{При }-1<a<0\text{ -- затухающие колебания (каждый шаг знак }a^k\text{ меняется)},\\
    &\text{При }a=0\text{ -- максимальная скорость сходимости},\\
    &\text{При }0\leq a<1\text{ -- монотонное сходящееся поведение},\\
    &\text{При }a=1\text{ -- нейтральная устойчивость},\\
    &\text{При }|a|>1\text{ -- неустойчивость}.
\end{align*}


Граница между отсутствием колебаний и их появлением при $a=0$:
$$
1-TK=0\Rightarrow K_{FB}=\frac{1}{TK_{CO}}\approx 0.454545
$$


Таким образом:
\begin{align*}
    &\text{При }K_{FB}=0\text{ -- выход повторяет }y_0,\\
    &\text{При }0<K_{FB}\leq0.454545\text{ -- монотонный выход},\\
    &\text{При }K_{FB}=0.454545\text{ -- оптимальный по быстродействию процесс},\\
    &\text{При }0.454545<K_{FB}<0.909091\text{ -- затухающие колебания},\\
    &\text{При }K_{FB}=0.909091\text{ -- максимальные незатухающие колебания},\\
    &\text{При }K_{FB}>0.909091\text{ -- растущие колебания}.
\end{align*}


Экстраполятор нулевого порядка (ZOH)
удерживает сигнал управления постоянным
на протяжении каждого шага дискретизации $T$.


ZOH переносит полюса системы из $s$-плоскости
в $z$-плоскость по преобразованию:
$$
z=e^{sT}
$$


Устойчивость сохраняется, когда все дискретные корни
(полюса) попадают внутрь единичного круга на $z$-плоскости.


Задержка уменьшает запас устойчивости системы
и делает ее чуствительной к выбору шага дискретизации $T$
-- вероятность смещения полюсов из
единичного круга увеличивается при увеличении $T$, следовательно
система с большей вероятностью может стать неустойчивой.


Выполним моделирование системы при различных $K_{FB}$.
\newpage


Положим $K_{FB}=0,y_0=0$:
\begin{figure}[H]
    \centering
    \includegraphics[scale=1]{k0_y0.png}
    \caption{Выход симуляции при $K_{FB}=0,y_0=0$}
    \label{fig:k0_y0}
\end{figure}


Положим $K_{FB}=0.3,y_0=0$:
\begin{figure}[H]
    \centering
    \includegraphics[scale=1]{k03_y0.png}
    \caption{Выход симуляции при $K_{FB}=0.3,y_0=0$}
    \label{fig:k03_y0}
\end{figure}


Положим $K_{FB}=1/\left( TK_{CO} \right),y_0=0$:
\begin{figure}[H]
    \centering
    \includegraphics[scale=1]{k045_y0.png}
    \caption{Выход симуляции при $K_{FB}=1/\left( TK_{CO} \right),y_0=0$}
    \label{fig:k045_y0}
\end{figure}


Положим $K_{FB}=0.6,y_0=0$:
\begin{figure}[H]
    \centering
    \includegraphics[scale=1]{k06_y0.png}
    \caption{Выход симуляции при $K_{FB}=0.6,y_0=0$}
    \label{fig:k06_y0}
\end{figure}


Положим $K_{FB}=2/\left( TK_{CO} \right),y_0=0$:
\begin{figure}[H]
    \centering
    \includegraphics[scale=1]{k09_y0.png}
    \caption{Выход симуляции при $K_{FB}=2/\left( TK_{CO} \right),y_0=0$}
    \label{fig:k09_y0}
\end{figure}


Положим $K_{FB}=1,y_0=0$:
\begin{figure}[H]
    \centering
    \includegraphics[scale=1]{k1_y0.png}
    \caption{Выход симуляции при $K_{FB}=1,y_0=0$}
    \label{fig:k1_y0}
\end{figure}


\section{Исследование устойчивости дискретных систем}
...


\section{Построение дискретных командных генераторов}
...


\section{Вывод}
...
\end{document}