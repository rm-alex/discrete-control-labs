\documentclass[a4paper,14pt]{extarticle}

\usepackage[T2A]{fontenc}
\usepackage[utf8]{inputenc}
\usepackage[english, russian]{babel}

\usepackage[left=30mm, right=10mm, top=20mm, bottom=20mm]{geometry}

\usepackage{tempora}
\usepackage{setspace}
\onehalfspacing

\usepackage{titlesec}
\titleformat{\section}[block]{\bfseries\centering\MakeUppercase}{\thesection.}{1em}{}
\titleformat{\subsection}[block]{\bfseries}{\thesubsection.}{1em}{}
\titleformat{\subsubsection}[block]{\bfseries}{\thesubsubsection.}{1em}{}

\renewcommand{\contentsname}{\hfill \textbf{СОДЕРЖАНИЕ} \hfill\null}

\usepackage{indentfirst}
\setlength{\parindent}{1.25cm}

\usepackage{amsmath, amsfonts, amssymb}
\usepackage{graphicx}
\usepackage{caption}
\usepackage{subcaption}
\usepackage{float}
\usepackage{tikz}
\usetikzlibrary{patterns}
\usepackage{cmap}
\usepackage{hyperref}
\usepackage{xcolor}
\usepackage{listings}

\definecolor{LightGray}{gray}{0.7}

\lstdefinestyle{code}{
    language=Python, % change if needed
    basicstyle=\small\ttfamily,
    numbers=left,
    numberstyle=\small\color{LightGray},
    stepnumber=1,
    numbersep=5pt,
    backgroundcolor=\color{white},
    showspaces=false,
    showstringspaces=false,
    showtabs=false,
    tabsize=4,
    captionpos=b,
    breaklines=true,
    breakatwhitespace=false,
    frame=single,
    rulecolor=\color{LightGray},
    linewidth=\linewidth,
    keywordstyle=\color{blue}\bfseries,
    commentstyle=\color{green!40!black},
    stringstyle=\color{violet},
    escapeinside={\%*}{*)},
    xleftmargin=10pt,
    xrightmargin=10pt,
    framexleftmargin=0pt,
    framexrightmargin=0pt
}
\lstset{style=code}

\hypersetup{
    colorlinks=true,
    linkcolor=blue,
    filecolor=magenta,
    urlcolor=cyan,
    pdftitle={ncs1},
    pdfauthor={Rumyantsev Alexey},
    pdfsubject={control},
    pdfkeywords={LaTeX, PDF},
    pdfpagemode=FullScreen,
}

\graphicspath{{src/images/}}

\begin{document}

\begin{titlepage}
    \begin{center}
        МИНИСТЕРСТВО НАУКИ И ВЫСШЕГО ОБРАЗОВАНИЯ РОССИЙСКОЙ ФЕДЕРАЦИИ\\
        \vspace*{2.5mm}
        Федеральное государственное автономное образовательное учреждение высшего образования
        «НАЦИОНАЛЬНЫЙ ИССЛЕДОВАТЕЛЬСКИЙ УНИВЕРСИТЕТ ИТМО»\\
        \vspace*{2.5mm}
        \textbf{ФАКУЛЬТЕТ СИСТЕМ УПРАВЛЕНИЯ И РОБОТОТЕХНИКИ}
        \vfill

        {\large ОТЧЕТ ПО ЛАБОРАТОРНОЙ РАБОТЕ №1}\\
        {\large по дисциплине}\\
        {\large\bfseries «ДИСКРЕТНЫЕ СИСТЕМЫ УПРАВЛЕНИЯ»}\\
        {\large на тему}\\
        {\large\bfseries «УСТОЙЧИВОСТЬ ДИСКРЕТНЫХ СИСТЕМ. ПОСТРОЕНИЕ ДИСКРЕТНЫХ КОМАНДНЫХ ГЕНЕРАТОРОВ»}\\
        Вариант 20
        \vfill

        \begin{flushright}
            Выполнил: студент гр. R3441\\
            Румянцев А. А.\medskip\\

            Проверил: преподаватель\\
            Краснов А. Ю.
        \end{flushright}
        \vfill

        Санкт-Петербург\\
        2025
    \end{center}
\end{titlepage}

\setcounter{page}{2}
\tableofcontents
\newpage


\section{Исследование влияния дискретного элемента на непрерывную систему}
Исходные данные для задания:
\begin{align*}
    &T=0.25\text{ с},\\
    &K_{CO}=8.8
\end{align*}


$T$ -- период дискретизации, $K_{CO}$
-- коэффициент передачи ОУ.


Схема симуляции:
\begin{figure}[H]
    \centering
    \includegraphics[scale=0.25]{sch1.png}
    \caption{Схема симуляции}
    \label{fig:sch1}
\end{figure}


Непрерывная система:
$$
y(t)=\int\limits_0^tu\left( \tau \right)d\,\tau
$$


Возьмем  $\tau\in\left[ t_k=kT,t_{k+1}=\left( k+1 \right)T \right)$ -- некоторый шаг дискретизации.
Приращение интеграла за этот интервал:
$$
y\left( \left( k+1 \right)T \right)-y\left( kT \right)=\int\limits_{kT}^{\left( k+1 \right)T}u\left( \tau \right)\,d\tau
$$


Обозначения:
$$
y_k=y\left( kT \right),\ u_k=u\left( kT \right)
$$


Дискретная система:
$$
y_{k+1}-y_k=Tu_k,
$$
$$
Tu_k=\int\limits_{kT}^{\left( k+1 \right)T}u_k\,d\tau\approx\int\limits_{kT}^{\left( k+1 \right)T}u\left( \tau \right)\,d\tau,
$$
$$
u_k=K_{CO}K_{FB}\,e_k=K\left( r_k-y_k \right),
$$
где $r_k$ -- вход системы.


Приведем систему к виду:
$$
y_{k+1}=ay_k+br_k,
$$
$$
y_{k+1}=y_k+TK\left( r_k-y_k \right)=\left( 1-TK \right)y_k+TKr_k
$$


Пусть вход -- единичная ступень $r_k=1\,\forall k\geq0$, тогда разностное уравнение:
$$
y_{k+1}=ay_k+b
$$


Соответствующая ему передаточная функция:
$$
W\left( z \right)=\frac{b}{z-a}
$$


Решение рекуррентного уравнения -- сумма общего $(h)$
и частного $(p)$ решений:
$$
y^{\left( h \right)}_{k+1}=ay^{\left( h \right)}_k\Rightarrow y^{\left( h \right)}_k=Ca^k,
$$
$$
y^{\left( p \right)}=ay^{\left( p \right)}+b\Rightarrow y^{\left( p \right)}\left( 1-a \right)=b\Rightarrow y^{\left( p \right)}=\frac{b}{1-a}=const.,
$$
$$
y_k=Ca^k+\frac{b}{1-a}
$$


При начальном условии $y_0=0$:
$$
Ca^0+\frac{b}{1-a}=0\Rightarrow C=-\frac{b}{1-a}
$$


Тогда:
$$
y_k=b\,\frac{1-a^k}{1-a}
$$


Переходный процесс описывается $a_k$.


При $k\to\infty,|a|<1:a^k\to0$, остается
установившееся значение:
$$
y_{\infty}=\frac{b}{1-a}
$$


Чем меньше $|a|$, тем быстрее убывает $a^k$ --
быстрее достигается установившееся значение.


Тогда, максимальная скорость сходимости будет при $a=0$.


Система устойчива при $|a|<1$:
$$
|1-TK|<1\Rightarrow 0<K_{FB}<\frac{2}{TK_{CO}}\approx0.909091
$$


Рассмотрим границы устойчивости.


Нейтральная устойчивость:
$$
a=1\Rightarrow 1-TK=1\Rightarrow K_{FB}=0
$$


При нейтральной устойчивости выход
повторяет начальное значение $y_0$,
задаваемое в интеграторе.


Колебательная устойчивость:
$$
a=-1\Rightarrow 1-TK=-1\Rightarrow K_{FB}=\frac{2}{TK_{CO}}\approx 0.909091
$$


При колебательной устойчивости
система будет максимально колебательной
без затухания.


Рассмотрим подробнее поведение системы при различных $a$:
\begin{align*}
    &\text{При }a=-1\text{ -- колебательная устойчивость},\\
    &\text{При }-1<a<0\text{ -- затухающие колебания (каждый шаг знак }a^k\text{ меняется)},\\
    &\text{При }a=0\text{ -- максимальная скорость сходимости},\\
    &\text{При }0\leq a<1\text{ -- монотонное сходящееся поведение},\\
    &\text{При }a=1\text{ -- нейтральная устойчивость},\\
    &\text{При }|a|>1\text{ -- неустойчивость}.
\end{align*}


Граница между отсутствием колебаний и их появлением при $a=0$:
$$
1-TK=0\Rightarrow K_{FB}=\frac{1}{TK_{CO}}\approx 0.454545
$$


Таким образом:
\begin{align*}
    &\text{При }K_{FB}=0\text{ -- выход повторяет }y_0,\\
    &\text{При }0<K_{FB}\leq0.454545\text{ -- монотонный выход},\\
    &\text{При }K_{FB}=0.454545\text{ -- оптимальный по быстродействию процесс},\\
    &\text{При }0.454545<K_{FB}<0.909091\text{ -- затухающие колебания},\\
    &\text{При }K_{FB}=0.909091\text{ -- максимальные незатухающие колебания},\\
    &\text{При }K_{FB}>0.909091\text{ -- растущие колебания}.
\end{align*}


Экстраполятор нулевого порядка (ZOH)
удерживает сигнал управления постоянным
на протяжении каждого шага дискретизации $T$
и переносит полюса системы из $s$-плоскости
в $z$-плоскость по преобразованию:
$$
z=e^{sT}
$$


Устойчивость сохраняется, когда все дискретные корни
(полюса) попадают внутрь единичного круга на $z$-плоскости.


Задержка уменьшает запас устойчивости системы
и делает ее чуствительной к выбору шага дискретизации $T$
-- вероятность смещения полюсов из
единичного круга увеличивается при увеличении $T$, следовательно
система с большей вероятностью может стать неустойчивой.


Выполним моделирование системы при различных $K_{FB}$.
\newpage


Положим $K_{FB}=0,y_0=0$:
\begin{figure}[H]
    \centering
    \includegraphics[scale=0.8]{k0_y0.png}
    \caption{Выход системы при $K_{FB}=0,y_0=0$}
    \label{fig:k0_y0}
\end{figure}


Положим $K_{FB}=0.3,y_0=0$:
\begin{figure}[H]
    \centering
    \includegraphics[scale=0.8]{k03_y0.png}
    \caption{Выход системы при $K_{FB}=0.3,y_0=0$}
    \label{fig:k03_y0}
\end{figure}


Положим $K_{FB}=1/\left( TK_{CO} \right),y_0=0$:
\begin{figure}[H]
    \centering
    \includegraphics[scale=0.8]{k045_y0.png}
    \caption{Выход системы при $K_{FB}=1/\left( TK_{CO} \right),y_0=0$}
    \label{fig:k045_y0}
\end{figure}


Положим $K_{FB}=0.6,y_0=0$:
\begin{figure}[H]
    \centering
    \includegraphics[scale=0.8]{k06_y0.png}
    \caption{Выход системы при $K_{FB}=0.6,y_0=0$}
    \label{fig:k06_y0}
\end{figure}


Положим $K_{FB}=2/\left( TK_{CO} \right),y_0=0$:
\begin{figure}[H]
    \centering
    \includegraphics[scale=0.8]{k09_y0.png}
    \caption{Выход системы при $K_{FB}=2/\left( TK_{CO} \right),y_0=0$}
    \label{fig:k09_y0}
\end{figure}


Положим $K_{FB}=1,y_0=0$:
\begin{figure}[H]
    \centering
    \includegraphics[scale=0.8]{k1_y0.png}
    \caption{Выход системы при $K_{FB}=1,y_0=0$}
    \label{fig:k1_y0}
\end{figure}


\section{Исследование устойчивости дискретных систем}
Исходные данные:
\begin{align*}
    &T=0.25\text{ с},\\
    &\sigma_1=\left\{ 1.59, 1.02 \right\},\\
    &\sigma_2=\left\{ -0.08, -0.69 \right\},\\
    &\sigma_3=\left\{ 0.29, -0.06 \right\},\\
    &\sigma_4=\left\{ 0.31j, -0.37j \right\},\\
    &\sigma_5=\left\{ -1.35\pm1.02j  \right\}
\end{align*}


Рассмотрим непрерывный ОУ:
$$
\ddot{y}=u,
$$
где $u(t)$ -- управляющее воздействие,
$y(t)$ -- выходная величина.


Положим:
$$
x_1=y,\ x_2=\dot{y}
$$


Тогда:
$$
\dot{x}_1=x_2,\ \dot{x}_2=u
$$


Система имеет вид (В-С-В):
$$
\dot{x}=A_\text{н}x+B_\text{н}u,\ y=Cx+Du,
$$
где:
$$
A_\text{н}=\begin{bmatrix}
    0&1\\0&0
\end{bmatrix},\ B_\text{н}=\begin{bmatrix}
    0\\1
\end{bmatrix},\ C=\begin{bmatrix}
    1&0
\end{bmatrix},\ D=0
$$


Приведем к дискретному виду:
$$
A=e^{A_\text{н}T}=\sum\limits_{i=0}^{k}\frac{A^{i}_\text{н}T^{i}}{i!},\ B=\int\limits_{0}^{T}e^{A_\text{н}\left( T-\tau \right)}\varphi\left( \tau \right)\,d\tau\cdot B_\text{н}=\sum\limits_{i=1}^{k}\frac{A^{i-1}_\text{н}T^{i}}{i!}B_\text{н}
$$


Рассмотрим $i=0,1,2$:
$$
A^0_\text{н}=\begin{bmatrix}
    1&0\\0&1
\end{bmatrix},\
A^1_\text{н}=\begin{bmatrix}
    0&1\\0&0
\end{bmatrix},\\ A^2_\text{н}=\begin{bmatrix}
    0&0\\0&0
\end{bmatrix}
$$


Достаточно взять $k=2$:
$$
A=I+A_\text{н}T+\frac{A^2_\text{н}T^2}{2}=\begin{bmatrix}
    1&T\\0&1
\end{bmatrix},
$$
$$
B=A^0_\text{н}TB_\text{н}+\frac{A^1_\text{н}T^2B_\text{н}}{2}=\begin{bmatrix}
    \frac{T^2}{2}\\T
\end{bmatrix}
$$


Дискретная модель:
$$
x_{k+1}=Ax_k+Bu_k,\ y_k=Cx_k,
$$
где:
$$
A=\begin{bmatrix}
    1&T\\0&1
\end{bmatrix},\ B=\begin{bmatrix}
    \frac{T^2}{2}\\T
\end{bmatrix},\ C=\begin{bmatrix}
    1&0
\end{bmatrix},\ D=0
$$


Зададим управляющее воздействие:
$$
u\left( k \right)=-Kx\left( k \right)=-\begin{bmatrix}
    k_1&k_2
\end{bmatrix}\begin{bmatrix}
    x_1\left( k \right)\\x_2\left( k \right)
\end{bmatrix}
$$


Матрица динамики замкнутой системы:
$$
F=A-BK=\begin{bmatrix}
\frac{-T^2k_1+2}{2}	&\frac{2T-T^2k_2}{2}\\
        -Tk_1	       &-Tk_2+1
\end{bmatrix}
$$


Замкнутая дискретная система:
$$
x_{k+1}=Fx_k
$$


Характеристический полином:
$$
\det{\left( \lambda I-F \right)}=\lambda^2+\lambda\left( T^2k_1/2+Tk_2-2 \right)+\left( T^2k_1/2-Tk_2+1 \right)=0
$$


Пусть желаемый полином:
$$
\left( \lambda-\lambda_1 \right)\left( \lambda-\lambda_2 \right)=\lambda^2-\left( \lambda_1+\lambda_2 \right)\lambda+\lambda_1\lambda_2
$$


Для спектра $\sigma_1=\left\{ 1.59, 1.02 \right\}$:
$$
\lambda^2-\left( 1.59+1.02 \right)\lambda+1.59\cdot1.02=\lambda^2-2.61\lambda+1.6218
$$


Следовательно:
$$
\begin{cases}
    T^2k_1/2+Tk_2-2 =-2.61,\\
     T^2k_1/2-Tk_2+1 =1.6218
\end{cases}\Rightarrow\begin{cases}
    T^2k_1/2+Tk_2 =-0.61,\\
     T^2k_1/2-Tk_2 =0.6218,
\end{cases}
$$
$$
\begin{cases}
    k_1=\frac{0.0118}{T^2}=0.1888,\\
    k_2=-\frac{0.6159}{T}=-2.4636
\end{cases}
$$


Проверим:
$$
\sigma\left( F \right)=\left\{ 1.02,1.59 \right\}
$$


Повторим для спектра $\sigma_2=\left\{ -0.08, -0.69 \right\}$:
$$
\lambda^2-\left( -0.08-0.69 \right)\lambda+0.08\cdot0.69=\lambda^2+0.77\lambda+0.0552,
$$
$$
\begin{cases}
    T^2k_1/2+Tk_2-2 =0.77,\\
     T^2k_1/2-Tk_2+1 =0.0552
\end{cases}\Rightarrow\begin{cases}
    T^2k_1/2+Tk_2 =2.77,\\
     T^2k_1/2-Tk_2=-0.9448,
\end{cases}
$$
$$
\begin{cases}
    k_1=\frac{1.8252}{T^2}=29.2032,\\
    k_2=\frac{0.6159}{T}=7.4296
\end{cases}
$$


Проверим:
$$
\sigma\left( F \right)=\left\{ -0.08,-0.69 \right\}
$$


Аналогично $\sigma_3=\left\{ 0.29, -0.06 \right\}$:
$$
\lambda^2-\left( 0.29-0.06 \right)\lambda-0.29\cdot0.06=\lambda^2-0.23\lambda-0.0174,
$$
$$
\begin{cases}
    T^2k_1/2+Tk_2-2 =-0.23,\\
     T^2k_1/2-Tk_2+1 =-0.0174
\end{cases}\Rightarrow\begin{cases}
    T^2k_1/2+Tk_2 =1.77,\\
     T^2k_1/2-Tk_2=-1.0174,
\end{cases}
$$
$$
\begin{cases}
    k_1=\frac{0.7526}{T^2}=12.0416,\\
    k_2=\frac{1.3937}{T}=5.5748,
\end{cases}\ \sigma\left( F \right)=\left\{ 0.29,-0.06 \right\}
$$


Аналогично $\sigma_4=\left\{ 0.31j, -0.37j \right\}$:
$$
\lambda^2-\left( 0.31j -0.37j \right)\lambda-0.31j\cdot0.37j=\lambda^2+0.06j\lambda+0.1147,
$$
$$
\begin{cases}
    T^2k_1/2+Tk_2-2 =0.06j,\\
     T^2k_1/2-Tk_2+1 =0.1147
\end{cases}\Rightarrow\begin{cases}
    T^2k_1/2+Tk_2 =2+0.06j,\\
     T^2k_1/2-Tk_2=-0.8853,
\end{cases}
$$
$$
\begin{cases}
    k_1=\frac{1.1147+0.06j}{T^2}=17.8352+0.96j,\\
    k_2=\frac{1.44265+0.03j}{T}=5.7706+0.12j,
\end{cases}\ \sigma\left( F \right)=\left\{ 0.31j,-0.37j \right\}
$$


Аналогично $\sigma_5=\left\{ -1.35\pm1.02j \right\}$:
$$
\lambda^2-\left( -1.35+1.02j-1.35-1.02j \right)\lambda+\left( -1.35+1.02j \right)\left( -1.35-1.02j \right),
$$
$$
\lambda^2+2.7\lambda+2.8629
$$
$$
\begin{cases}
    T^2k_1/2+Tk_2-2 =2.7,\\
     T^2k_1/2-Tk_2+1 =2.8629
\end{cases}\Rightarrow\begin{cases}
    T^2k_1/2+Tk_2 =4.7,\\
     T^2k_1/2-Tk_2=1.8629,
\end{cases}
$$
$$
\begin{cases}
    k_1=\frac{6.5629}{T^2}=105.0064,\\
    k_2=\frac{1.41855}{T}=5.6742,
\end{cases}\ \sigma\left( F \right)=\left\{ -1.35\pm1.02j \right\}
$$


Выполним моделирование систем при $y\left( 0 \right)=1,\dot{y}\left( 0 \right)=0$.


\begin{figure}[H]
    \centering
    \includegraphics[scale=0.8]{2_s1.png}
    \caption{Выход системы при $\sigma_1=\left\{ 1.02,1.59 \right\}$}
    \label{fig:2_s1}
\end{figure}


\begin{figure}[H]
    \centering
    \includegraphics[scale=0.8]{2_s2.png}
    \caption{Выход системы при $\sigma_2=\left\{ -0.08, -0.69  \right\}$}
    \label{fig:2_s2}
\end{figure}


\begin{figure}[H]
    \centering
    \includegraphics[scale=0.8]{2_s3.png}
    \caption{Выход системы при $\sigma_3=\left\{ 0.29,-0.06  \right\}$}
    \label{fig:2_s3}
\end{figure}


\begin{figure}[H]
    \centering
    \includegraphics[scale=0.8]{2_s4.png}
    \caption{Выход системы при $\sigma_4=\left\{ 0.31j,-0.37j  \right\}$}
    \label{fig:2_s4}
\end{figure}


\begin{figure}[H]
    \centering
    \includegraphics[scale=0.8]{2_s5.png}
    \caption{Выход системы при $\sigma_5=\left\{ -1.35\pm1.02j \right\}$}
    \label{fig:2_s5}
\end{figure}


Когда вещественная часть собственных чисел по модулю больше 1, то система расходится.


В обратном случае система сходится к устоявшемуся состоянию.


\section{Построение дискретных командных генераторов}
Исходные данные:
\begin{align*}
    &T=0.25\text{ с},\\
    &A=-8.5,\\
    &\omega=0.7,\\
    &f=0.4\sin{\left( 2kT \right)}+2.5\cos{\left( 6kT \right)},
\end{align*}
где $f$ -- внешнее возмущение.


Синтезируем командный генератор гармонического сигнала:
$$
g\left( k \right)=A\sin{\left( kT\omega \right)}
$$


Выберем в качестве первой координаты вектора
состояния дискретного генератора сам сигнал:
$$
\xi_1\left( k \right)=g\left( k \right)
$$


Возьмем первую разность (аналог производной) и выберем
вторую переменную вектора
состояния в виде этой разности:
\begin{align*}
    \xi_2\left( k \right)&=\xi_1\left( k+1 \right)=
    A\sin{\left( kT\omega+T\omega \right)}=\\
    &=A\left[ \sin{\left( kT\omega \right)}\cos{\left( T\omega \right)}+\cos{\left( kT\omega \right)}\sin{\left( T\omega \right)} \right]=\\
    &=\xi_1\left( k \right)\cos{\left( T\omega \right)}+A\cos{\left( kT\omega \right)}\sin{\left( T\omega \right)}
\end{align*}


Возьмем вторую разность и выберем
третью переменную вектора состояния
в виде этой разности:
\begin{align*}
    \xi_3\left( k \right)&=\xi_2\left( k+1 \right)=
    \xi_2\left( k \right)\cos{\left( T\omega \right)}+
    A\cos{\left( kT\omega+T\omega \right)}\sin{\left( T\omega \right)}=\\
    &=\xi_2\left( k \right)\cos{\left( T\omega \right)}+A\sin{\left( T\omega \right)}
    \left[ \cos{\left( kT\omega \right)}\cos{\left( T\omega \right)}-\sin{\left( kT\omega \right)}\sin{\left( T\omega \right)} \right]=\\
    &=\xi_2\left( k \right)\cos{\left( T\omega \right)}+\xi_2\left( k \right)\cos{\left( T\omega \right)}-\xi_1\left( k \right)\cos^2{\left( T\omega \right)}-\xi_1\left( k \right)\sin^2{\left( T\omega \right)}=\\
    &=2\xi_2\left( k \right)\cos{\left( T\omega \right)}-\xi_1\left( k \right)
\end{align*}


Третья строчка из второй получается путем выражения второго слагаемого
из выражения для $\xi_2\left( k \right)$.


% Чтобы задать состояние системы,
% необходимо знать значение $g\left( t \right)$
% и его скорость изменения $\dot{g}\left( t \right)$.


Достаточно двух разностей -- для
вычисления $g\left( k+2 \right)$
достаточно знать $g\left( k+1 \right)$ и $g\left( k \right)$
(разностное уравнение второго порядка).


Система выражений:
$$
\begin{cases}
    \xi_1\left( k+1 \right)=\xi_2\left( k \right),\\
    \xi_2\left( k+1 \right)=2\xi_2\left( k \right)\cos{\left( T\omega \right)}-\xi_1\left( k \right),\\
    g\left( k \right)=\xi_1\left( k \right)
\end{cases}
$$


Вектор начальных условий при $k=0$:
$$
\xi\left( 0 \right)=\begin{bmatrix}
    0\\A\sin{\left( T\omega \right)}
\end{bmatrix}
$$


Командный генератор:
$$
\begin{bmatrix}
    \xi_1\left( k+1 \right)\\
    \xi_2\left( k+1 \right)
\end{bmatrix}=\begin{bmatrix}
    0&1\\-1&2\cos{\left( T\omega \right)}
\end{bmatrix}\begin{bmatrix}
    \xi_1\left( k \right)\\
    \xi_2\left( k \right)
\end{bmatrix}=\Gamma_d\,\xi\left( k \right),
$$
$$
g\left( k \right)=\begin{bmatrix}
    1&0
\end{bmatrix}\begin{bmatrix}
    \xi_1\left( k \right)\\
    \xi_2\left( k \right)
\end{bmatrix}=H\xi\left( k \right)
$$


Схема моделирования командного генератора:
\begin{figure}[H]
    \centering
    \includegraphics[scale=0.55]{3_1.png}
    \caption{Схема командного генератора}
    \label{fig:3_1}
\end{figure}


Выполним моделирование командного генератора.
\begin{figure}[H]
    \centering
    \includegraphics[scale=0.8]{3_11.png}
    \caption{Выход генератора}
    \label{fig:3_11}
\end{figure}


Синтезируем дискретную математическую модель возмущения
<<вход-состояние-выход>> с параметрами:
$$
g\left( k \right)=0.4\sin{\left( 2kT \right)}+2.5\cos{\left( 6kT \right)}
$$


Алгоритм аналогичный:
$$
\xi_1\left( k \right)=g\left( k \right)
$$


\section{Вывод}
...
\end{document}